% Options for packages loaded elsewhere
\PassOptionsToPackage{unicode}{hyperref}
\PassOptionsToPackage{hyphens}{url}
\PassOptionsToPackage{dvipsnames,svgnames,x11names}{xcolor}
%
\documentclass[
  letterpaper,
  DIV=11,
  numbers=noendperiod]{scrartcl}

\usepackage{amsmath,amssymb}
\usepackage{iftex}
\ifPDFTeX
  \usepackage[T1]{fontenc}
  \usepackage[utf8]{inputenc}
  \usepackage{textcomp} % provide euro and other symbols
\else % if luatex or xetex
  \usepackage{unicode-math}
  \defaultfontfeatures{Scale=MatchLowercase}
  \defaultfontfeatures[\rmfamily]{Ligatures=TeX,Scale=1}
\fi
\usepackage{lmodern}
\ifPDFTeX\else  
    % xetex/luatex font selection
\fi
% Use upquote if available, for straight quotes in verbatim environments
\IfFileExists{upquote.sty}{\usepackage{upquote}}{}
\IfFileExists{microtype.sty}{% use microtype if available
  \usepackage[]{microtype}
  \UseMicrotypeSet[protrusion]{basicmath} % disable protrusion for tt fonts
}{}
\makeatletter
\@ifundefined{KOMAClassName}{% if non-KOMA class
  \IfFileExists{parskip.sty}{%
    \usepackage{parskip}
  }{% else
    \setlength{\parindent}{0pt}
    \setlength{\parskip}{6pt plus 2pt minus 1pt}}
}{% if KOMA class
  \KOMAoptions{parskip=half}}
\makeatother
\usepackage{xcolor}
\setlength{\emergencystretch}{3em} % prevent overfull lines
\setcounter{secnumdepth}{5}
% Make \paragraph and \subparagraph free-standing
\ifx\paragraph\undefined\else
  \let\oldparagraph\paragraph
  \renewcommand{\paragraph}[1]{\oldparagraph{#1}\mbox{}}
\fi
\ifx\subparagraph\undefined\else
  \let\oldsubparagraph\subparagraph
  \renewcommand{\subparagraph}[1]{\oldsubparagraph{#1}\mbox{}}
\fi

\usepackage{color}
\usepackage{fancyvrb}
\newcommand{\VerbBar}{|}
\newcommand{\VERB}{\Verb[commandchars=\\\{\}]}
\DefineVerbatimEnvironment{Highlighting}{Verbatim}{commandchars=\\\{\}}
% Add ',fontsize=\small' for more characters per line
\usepackage{framed}
\definecolor{shadecolor}{RGB}{241,243,245}
\newenvironment{Shaded}{\begin{snugshade}}{\end{snugshade}}
\newcommand{\AlertTok}[1]{\textcolor[rgb]{0.68,0.00,0.00}{#1}}
\newcommand{\AnnotationTok}[1]{\textcolor[rgb]{0.37,0.37,0.37}{#1}}
\newcommand{\AttributeTok}[1]{\textcolor[rgb]{0.40,0.45,0.13}{#1}}
\newcommand{\BaseNTok}[1]{\textcolor[rgb]{0.68,0.00,0.00}{#1}}
\newcommand{\BuiltInTok}[1]{\textcolor[rgb]{0.00,0.23,0.31}{#1}}
\newcommand{\CharTok}[1]{\textcolor[rgb]{0.13,0.47,0.30}{#1}}
\newcommand{\CommentTok}[1]{\textcolor[rgb]{0.37,0.37,0.37}{#1}}
\newcommand{\CommentVarTok}[1]{\textcolor[rgb]{0.37,0.37,0.37}{\textit{#1}}}
\newcommand{\ConstantTok}[1]{\textcolor[rgb]{0.56,0.35,0.01}{#1}}
\newcommand{\ControlFlowTok}[1]{\textcolor[rgb]{0.00,0.23,0.31}{#1}}
\newcommand{\DataTypeTok}[1]{\textcolor[rgb]{0.68,0.00,0.00}{#1}}
\newcommand{\DecValTok}[1]{\textcolor[rgb]{0.68,0.00,0.00}{#1}}
\newcommand{\DocumentationTok}[1]{\textcolor[rgb]{0.37,0.37,0.37}{\textit{#1}}}
\newcommand{\ErrorTok}[1]{\textcolor[rgb]{0.68,0.00,0.00}{#1}}
\newcommand{\ExtensionTok}[1]{\textcolor[rgb]{0.00,0.23,0.31}{#1}}
\newcommand{\FloatTok}[1]{\textcolor[rgb]{0.68,0.00,0.00}{#1}}
\newcommand{\FunctionTok}[1]{\textcolor[rgb]{0.28,0.35,0.67}{#1}}
\newcommand{\ImportTok}[1]{\textcolor[rgb]{0.00,0.46,0.62}{#1}}
\newcommand{\InformationTok}[1]{\textcolor[rgb]{0.37,0.37,0.37}{#1}}
\newcommand{\KeywordTok}[1]{\textcolor[rgb]{0.00,0.23,0.31}{#1}}
\newcommand{\NormalTok}[1]{\textcolor[rgb]{0.00,0.23,0.31}{#1}}
\newcommand{\OperatorTok}[1]{\textcolor[rgb]{0.37,0.37,0.37}{#1}}
\newcommand{\OtherTok}[1]{\textcolor[rgb]{0.00,0.23,0.31}{#1}}
\newcommand{\PreprocessorTok}[1]{\textcolor[rgb]{0.68,0.00,0.00}{#1}}
\newcommand{\RegionMarkerTok}[1]{\textcolor[rgb]{0.00,0.23,0.31}{#1}}
\newcommand{\SpecialCharTok}[1]{\textcolor[rgb]{0.37,0.37,0.37}{#1}}
\newcommand{\SpecialStringTok}[1]{\textcolor[rgb]{0.13,0.47,0.30}{#1}}
\newcommand{\StringTok}[1]{\textcolor[rgb]{0.13,0.47,0.30}{#1}}
\newcommand{\VariableTok}[1]{\textcolor[rgb]{0.07,0.07,0.07}{#1}}
\newcommand{\VerbatimStringTok}[1]{\textcolor[rgb]{0.13,0.47,0.30}{#1}}
\newcommand{\WarningTok}[1]{\textcolor[rgb]{0.37,0.37,0.37}{\textit{#1}}}

\providecommand{\tightlist}{%
  \setlength{\itemsep}{0pt}\setlength{\parskip}{0pt}}\usepackage{longtable,booktabs,array}
\usepackage{calc} % for calculating minipage widths
% Correct order of tables after \paragraph or \subparagraph
\usepackage{etoolbox}
\makeatletter
\patchcmd\longtable{\par}{\if@noskipsec\mbox{}\fi\par}{}{}
\makeatother
% Allow footnotes in longtable head/foot
\IfFileExists{footnotehyper.sty}{\usepackage{footnotehyper}}{\usepackage{footnote}}
\makesavenoteenv{longtable}
\usepackage{graphicx}
\makeatletter
\def\maxwidth{\ifdim\Gin@nat@width>\linewidth\linewidth\else\Gin@nat@width\fi}
\def\maxheight{\ifdim\Gin@nat@height>\textheight\textheight\else\Gin@nat@height\fi}
\makeatother
% Scale images if necessary, so that they will not overflow the page
% margins by default, and it is still possible to overwrite the defaults
% using explicit options in \includegraphics[width, height, ...]{}
\setkeys{Gin}{width=\maxwidth,height=\maxheight,keepaspectratio}
% Set default figure placement to htbp
\makeatletter
\def\fps@figure{htbp}
\makeatother
% definitions for citeproc citations
\NewDocumentCommand\citeproctext{}{}
\NewDocumentCommand\citeproc{mm}{%
  \begingroup\def\citeproctext{#2}\cite{#1}\endgroup}
\makeatletter
 % allow citations to break across lines
 \let\@cite@ofmt\@firstofone
 % avoid brackets around text for \cite:
 \def\@biblabel#1{}
 \def\@cite#1#2{{#1\if@tempswa , #2\fi}}
\makeatother
\newlength{\cslhangindent}
\setlength{\cslhangindent}{1.5em}
\newlength{\csllabelwidth}
\setlength{\csllabelwidth}{3em}
\newenvironment{CSLReferences}[2] % #1 hanging-indent, #2 entry-spacing
 {\begin{list}{}{%
  \setlength{\itemindent}{0pt}
  \setlength{\leftmargin}{0pt}
  \setlength{\parsep}{0pt}
  % turn on hanging indent if param 1 is 1
  \ifodd #1
   \setlength{\leftmargin}{\cslhangindent}
   \setlength{\itemindent}{-1\cslhangindent}
  \fi
  % set entry spacing
  \setlength{\itemsep}{#2\baselineskip}}}
 {\end{list}}
\usepackage{calc}
\newcommand{\CSLBlock}[1]{\hfill\break\parbox[t]{\linewidth}{\strut\ignorespaces#1\strut}}
\newcommand{\CSLLeftMargin}[1]{\parbox[t]{\csllabelwidth}{\strut#1\strut}}
\newcommand{\CSLRightInline}[1]{\parbox[t]{\linewidth - \csllabelwidth}{\strut#1\strut}}
\newcommand{\CSLIndent}[1]{\hspace{\cslhangindent}#1}

\KOMAoption{captions}{tableheading,figureheading}
\makeatletter
\@ifpackageloaded{caption}{}{\usepackage{caption}}
\AtBeginDocument{%
\ifdefined\contentsname
  \renewcommand*\contentsname{Table of contents}
\else
  \newcommand\contentsname{Table of contents}
\fi
\ifdefined\listfigurename
  \renewcommand*\listfigurename{List of Figures}
\else
  \newcommand\listfigurename{List of Figures}
\fi
\ifdefined\listtablename
  \renewcommand*\listtablename{List of Tables}
\else
  \newcommand\listtablename{List of Tables}
\fi
\ifdefined\figurename
  \renewcommand*\figurename{Figure}
\else
  \newcommand\figurename{Figure}
\fi
\ifdefined\tablename
  \renewcommand*\tablename{Table}
\else
  \newcommand\tablename{Table}
\fi
}
\@ifpackageloaded{float}{}{\usepackage{float}}
\floatstyle{ruled}
\@ifundefined{c@chapter}{\newfloat{codelisting}{h}{lop}}{\newfloat{codelisting}{h}{lop}[chapter]}
\floatname{codelisting}{Listing}
\newcommand*\listoflistings{\listof{codelisting}{List of Listings}}
\makeatother
\makeatletter
\makeatother
\makeatletter
\@ifpackageloaded{caption}{}{\usepackage{caption}}
\@ifpackageloaded{subcaption}{}{\usepackage{subcaption}}
\makeatother
\ifLuaTeX
  \usepackage{selnolig}  % disable illegal ligatures
\fi
\usepackage{bookmark}

\IfFileExists{xurl.sty}{\usepackage{xurl}}{} % add URL line breaks if available
\urlstyle{same} % disable monospaced font for URLs
\hypersetup{
  pdftitle={Supplement Material},
  colorlinks=true,
  linkcolor={blue},
  filecolor={Maroon},
  citecolor={Blue},
  urlcolor={Blue},
  pdfcreator={LaTeX via pandoc}}

\title{Supplement Material}
\usepackage{etoolbox}
\makeatletter
\providecommand{\subtitle}[1]{% add subtitle to \maketitle
  \apptocmd{\@title}{\par {\large #1 \par}}{}{}
}
\makeatother
\subtitle{phoenix: R package and Python module for calculating the
Phoenix Pediatric Sepsis Score and Criteria}
\author{}
\date{}

\begin{document}
\maketitle

\renewcommand*\contentsname{Table of contents}
{
\hypersetup{linkcolor=}
\setcounter{tocdepth}{3}
\tableofcontents
}
\newpage{}

Methods for applying the Phoenix organ dysfunction scoring and criteria
in R and Python have been packaged and published. Example SQL queries
(SQLite) are also presented here.

\section{Installing Software}\label{installing-software}

\subsection{R}\label{r}

To install the R package

\begin{enumerate}
\def\labelenumi{\arabic{enumi}.}
\tightlist
\item
  Install R
\item
  Within R call
  \texttt{install.packages(\textquotesingle{}phoenix\textquotesingle{})}
  to get the current released version.
\end{enumerate}

Install the development version of \texttt{phoenix} directly from GitHub
via the \href{https://github.com/r-lib/remotes/}{\texttt{remotes}}
package{[}1{]}:

\begin{Shaded}
\begin{Highlighting}[]
\ControlFlowTok{if}\NormalTok{ (}\SpecialCharTok{!}\NormalTok{(}\StringTok{"remotes"} \SpecialCharTok{\%in\%} \FunctionTok{rownames}\NormalTok{(}\FunctionTok{installed.packages}\NormalTok{()))) \{}
  \FunctionTok{warning}\NormalTok{(}\StringTok{"installing remotes from https://cran.rstudio.com"}\NormalTok{)}
  \FunctionTok{install.packages}\NormalTok{(}\StringTok{"remotes"}\NormalTok{, }\AttributeTok{repo =} \StringTok{"https://cran.rstudio.com"}\NormalTok{)}
\NormalTok{\}}

\NormalTok{remotes}\SpecialCharTok{::}\FunctionTok{install\_github}\NormalTok{(}\StringTok{"cu{-}dbmi{-}peds/phoenix"}\NormalTok{)}
\end{Highlighting}
\end{Shaded}

\emph{NOTE:} If you are working on a Windows machine, you will need to
download and install
\href{https://cran.r-project.org/bin/windows/Rtools/}{\texttt{Rtools}}.

\subsection{Python}\label{python}

To install the Python module use \texttt{pip}.

\begin{Shaded}
\begin{Highlighting}[]
\NormalTok{pip install phoenix}\OperatorTok{{-}}\NormalTok{sepsis}
\end{Highlighting}
\end{Shaded}

\section{Example Data}\label{example-data}

A small example data set has been provided in the phoenix R package and
the phoenix-sepsis. In R the data set, \texttt{sepsis}, is a provided as
a data.frame. In Python, it is provided as a .csv file with instructions
for loading it into a pandas DataFrame. For the SQL examples we'll use a
in-memory SQLite database and load the data into a table called
\texttt{sepsis.}

The data set consists of 20 observations of 27 variables. The column
names and ordering are consistent between the R package and Python
module.

\begin{longtable}[]{@{}
  >{\raggedright\arraybackslash}p{(\columnwidth - 2\tabcolsep) * \real{0.3333}}
  >{\raggedright\arraybackslash}p{(\columnwidth - 2\tabcolsep) * \real{0.6667}}@{}}
\toprule\noalign{}
\begin{minipage}[b]{\linewidth}\raggedright
Column Name
\end{minipage} & \begin{minipage}[b]{\linewidth}\raggedright
Description
\end{minipage} \\
\midrule\noalign{}
\endhead
\bottomrule\noalign{}
\endlastfoot
\texttt{pid} & patient identification number \\
\texttt{age} & age in months \\
\texttt{fio2} & fraction of inspired oxygen \\
\texttt{pao2} & partial pressure of oxygen in arterial blood (mmHg) \\
\texttt{spo2} & pulse oximetry \\
\texttt{vent} & indicator for invasive mechanical ventilation \\
\texttt{gcs\_total} & total Glasgow Coma Scale \\
\texttt{pupil} & character vector reporting if pupils are reactive or
fixed. \\
\texttt{platelets} & platelets measured in 1,000 / \(\mu\)L \\
\texttt{inr} & international normalized ratio \\
\texttt{d\_dimer} & D-dimer; units of mg/L FEU \\
\texttt{fibrinogen} & units of mg/dL \\
\texttt{dbp} & diagnostic blood pressure (mmHg) \\
\texttt{sbp} & systolic blood pressure (mmHg) \\
\texttt{lactate} & units of mmol/L \\
\texttt{dobutamine} & indicator for receiving systemic dobutamine \\
\texttt{dopamine} & indicator for receiving systemic dopamine \\
\texttt{epinephrine} & indicator for receiving systemic epinephrine \\
\texttt{milrinone} & indicator for receiving systemic milrinone \\
\texttt{norepinephrine} & indicator for receiving systemic
norepinephrine \\
\texttt{vasopressin} & indicator for receiving systemic vasopressin \\
\texttt{glucose} & units of mg/dL \\
\texttt{anc} & units of 1,000 cells per cubic millimeter \\
\texttt{alc} & units of 1,000 cells per cubic millimeter \\
\texttt{creatinine} & units of mg/dL \\
\texttt{bilirubin} & units of mg/dL \\
\texttt{alt} & units of IU/L \\
\end{longtable}

\subsection{R}\label{r-1}

The \texttt{sepsis} data set is lazyloaded when the package namespace is
loaded and attached. No other packages are needed to run the R examples
in this supplement.

\begin{Shaded}
\begin{Highlighting}[]
\FunctionTok{library}\NormalTok{(phoenix)}
\FunctionTok{str}\NormalTok{(sepsis)}
\DocumentationTok{\#\# \textquotesingle{}data.frame\textquotesingle{}:    20 obs. of  27 variables:}
\DocumentationTok{\#\#  $ pid           : int  1 2 3 4 5 6 7 8 9 10 ...}
\DocumentationTok{\#\#  $ age           : num  0.06 201.7 20.8 192.5 214.4 ...}
\DocumentationTok{\#\#  $ fio2          : num  0.75 0.75 1 NA NA 0.6 0.5 0.3 0.65 0.8 ...}
\DocumentationTok{\#\#  $ pao2          : num  NA 75.3 49.5 NA 38.7 69.9 NA NA 51 NA ...}
\DocumentationTok{\#\#  $ spo2          : int  99 95 NA NA 95 88 31 97 82 76 ...}
\DocumentationTok{\#\#  $ vent          : int  1 1 1 0 0 1 1 1 1 1 ...}
\DocumentationTok{\#\#  $ gcs\_total     : int  NA 5 15 14 NA 3 NA 15 3 3 ...}
\DocumentationTok{\#\#  $ pupil         : chr  "" "both{-}reactive" "both{-}reactive" "" ...}
\DocumentationTok{\#\#  $ platelets     : int  199 243 49 NA 393 86 65 215 101 292 ...}
\DocumentationTok{\#\#  $ inr           : num  1.46 1.18 1.6 1.3 NA 1.23 3.1 0.97 1.08 NA ...}
\DocumentationTok{\#\#  $ d\_dimer       : num  NA 2.45 NA 2.82 NA 4.72 NA 5.15 7.71 NA ...}
\DocumentationTok{\#\#  $ fibrinogen    : int  180 311 309 220 NA 270 94 489 456 NA ...}
\DocumentationTok{\#\#  $ dbp           : int  40 60 87 57 57 79 11 66 51 58 ...}
\DocumentationTok{\#\#  $ sbp           : int  53 90 233 104 101 119 14 112 117 84 ...}
\DocumentationTok{\#\#  $ lactate       : num  NA 3.32 1 NA NA 1.15 NA NA 8.1 NA ...}
\DocumentationTok{\#\#  $ dobutamine    : int  1 0 0 0 0 0 0 0 0 0 ...}
\DocumentationTok{\#\#  $ dopamine      : int  1 1 1 0 0 1 0 0 0 0 ...}
\DocumentationTok{\#\#  $ epinephrine   : int  1 0 0 0 0 0 1 0 1 0 ...}
\DocumentationTok{\#\#  $ milrinone     : int  1 0 0 0 0 0 1 0 1 0 ...}
\DocumentationTok{\#\#  $ norepinephrine: int  0 1 0 0 0 0 0 0 1 0 ...}
\DocumentationTok{\#\#  $ vasopressin   : int  0 0 0 0 0 0 1 0 1 0 ...}
\DocumentationTok{\#\#  $ glucose       : num  NA 110 93 110 NA 147 NA 100 264 93 ...}
\DocumentationTok{\#\#  $ anc           : num  NA 14.22 2.21 3.18 NA ...}
\DocumentationTok{\#\#  $ alc           : num  NA 2.22 0.19 0.645 NA ...}
\DocumentationTok{\#\#  $ creatinine    : num  1.03 0.51 0.33 0.31 0.52 0.77 1.47 0.58 1.23 0.18 ...}
\DocumentationTok{\#\#  $ bilirubin     : num  NA 0.2 0.8 8.5 NA 1.2 1.7 0.5 21.1 NA ...}
\DocumentationTok{\#\#  $ alt           : int  36 32 182 21 NA 15 3664 50 151 NA ...}
\end{Highlighting}
\end{Shaded}

\subsection{Python}\label{python-1}

Along with with \texttt{phoenix} module, you will need \texttt{numpy},
\texttt{pandas}, and \texttt{importlib.resources} to run all the Python
examples in this supplement.

\begin{Shaded}
\begin{Highlighting}[]
\ImportTok{import}\NormalTok{ numpy }\ImportTok{as}\NormalTok{ np}
\ImportTok{import}\NormalTok{ pandas }\ImportTok{as}\NormalTok{ pd}
\ImportTok{import}\NormalTok{ importlib.resources}
\ImportTok{import}\NormalTok{ phoenix }\ImportTok{as}\NormalTok{ phx}

\NormalTok{path }\OperatorTok{=}\NormalTok{ importlib.resources.files(}\StringTok{\textquotesingle{}phoenix\textquotesingle{}}\NormalTok{)}
\NormalTok{sepsis }\OperatorTok{=}\NormalTok{ pd.read\_csv(path.joinpath(}\StringTok{\textquotesingle{}data\textquotesingle{}}\NormalTok{).joinpath(}\StringTok{\textquotesingle{}sepsis.csv\textquotesingle{}}\NormalTok{))}

\BuiltInTok{print}\NormalTok{(sepsis.info())}
\CommentTok{\#\# \textless{}class \textquotesingle{}pandas.core.frame.DataFrame\textquotesingle{}\textgreater{}}
\CommentTok{\#\# RangeIndex: 20 entries, 0 to 19}
\CommentTok{\#\# Data columns (total 27 columns):}
\CommentTok{\#\#  \#   Column          Non{-}Null Count  Dtype  }
\CommentTok{\#\# {-}{-}{-}  {-}{-}{-}{-}{-}{-}          {-}{-}{-}{-}{-}{-}{-}{-}{-}{-}{-}{-}{-}{-}  {-}{-}{-}{-}{-}  }
\CommentTok{\#\#  0   pid             20 non{-}null     int64  }
\CommentTok{\#\#  1   age             20 non{-}null     float64}
\CommentTok{\#\#  2   fio2            16 non{-}null     float64}
\CommentTok{\#\#  3   pao2            9 non{-}null      float64}
\CommentTok{\#\#  4   spo2            15 non{-}null     float64}
\CommentTok{\#\#  5   vent            20 non{-}null     int64  }
\CommentTok{\#\#  6   gcs\_total       9 non{-}null      float64}
\CommentTok{\#\#  7   pupil           8 non{-}null      object }
\CommentTok{\#\#  8   platelets       16 non{-}null     float64}
\CommentTok{\#\#  9   inr             15 non{-}null     float64}
\CommentTok{\#\#  10  d\_dimer         8 non{-}null      float64}
\CommentTok{\#\#  11  fibrinogen      13 non{-}null     float64}
\CommentTok{\#\#  12  dbp             20 non{-}null     int64  }
\CommentTok{\#\#  13  sbp             20 non{-}null     int64  }
\CommentTok{\#\#  14  lactate         7 non{-}null      float64}
\CommentTok{\#\#  15  dobutamine      20 non{-}null     int64  }
\CommentTok{\#\#  16  dopamine        20 non{-}null     int64  }
\CommentTok{\#\#  17  epinephrine     20 non{-}null     int64  }
\CommentTok{\#\#  18  milrinone       20 non{-}null     int64  }
\CommentTok{\#\#  19  norepinephrine  20 non{-}null     int64  }
\CommentTok{\#\#  20  vasopressin     20 non{-}null     int64  }
\CommentTok{\#\#  21  glucose         12 non{-}null     float64}
\CommentTok{\#\#  22  anc             11 non{-}null     float64}
\CommentTok{\#\#  23  alc             11 non{-}null     float64}
\CommentTok{\#\#  24  creatinine      18 non{-}null     float64}
\CommentTok{\#\#  25  bilirubin       14 non{-}null     float64}
\CommentTok{\#\#  26  alt             14 non{-}null     float64}
\CommentTok{\#\# dtypes: float64(16), int64(10), object(1)}
\CommentTok{\#\# memory usage: 4.3+ KB}
\CommentTok{\#\# None}
\BuiltInTok{print}\NormalTok{(sepsis.head())}
\CommentTok{\#\#    pid     age  fio2  pao2  spo2  ...     anc    alc creatinine  bilirubin    alt}
\CommentTok{\#\# 0    1    0.06  0.75   NaN  99.0  ...     NaN    NaN       1.03        NaN   36.0}
\CommentTok{\#\# 1    2  201.70  0.75  75.3  95.0  ...  14.220  2.220       0.51        0.2   32.0}
\CommentTok{\#\# 2    3   20.80  1.00  49.5   NaN  ...   2.210  0.190       0.33        0.8  182.0}
\CommentTok{\#\# 3    4  192.50   NaN   NaN   NaN  ...   3.184  0.645       0.31        8.5   21.0}
\CommentTok{\#\# 4    5  214.40   NaN  38.7  95.0  ...     NaN    NaN       0.52        NaN    NaN}
\CommentTok{\#\# }
\CommentTok{\#\# [5 rows x 27 columns]}
\end{Highlighting}
\end{Shaded}

\subsection{SQL}\label{sql}

The example SQL queries are done in a in-memory SQLite instance set up
within R. The following code chunk sets up the database with the example
sepsis data loaded into a table of the same name.

\begin{Shaded}
\begin{Highlighting}[]
\FunctionTok{library}\NormalTok{(odbc)}
\FunctionTok{library}\NormalTok{(DBI)}
\FunctionTok{library}\NormalTok{(RSQLite)}
\FunctionTok{library}\NormalTok{(phoenix)}
\NormalTok{con }\OtherTok{\textless{}{-}} \FunctionTok{dbConnect}\NormalTok{(}\AttributeTok{drv =}\NormalTok{ RSQLite}\SpecialCharTok{::}\FunctionTok{SQLite}\NormalTok{(), }\AttributeTok{dbname =} \StringTok{":memory:"}\NormalTok{)}
\FunctionTok{dbWriteTable}\NormalTok{(}\AttributeTok{conn =}\NormalTok{ con, }\AttributeTok{name =} \StringTok{"sepsis"}\NormalTok{, }\AttributeTok{value =}\NormalTok{ sepsis)}
\end{Highlighting}
\end{Shaded}

The structure of the table:

\begin{Shaded}
\begin{Highlighting}[]
\NormalTok{PRAGMA table\_info(sepsis);}
\end{Highlighting}
\end{Shaded}

\begin{longtable}[]{@{}lllrlr@{}}
\caption{Displaying records 1 - 10}\tabularnewline
\toprule\noalign{}
cid & name & type & notnull & dflt\_value & pk \\
\midrule\noalign{}
\endfirsthead
\toprule\noalign{}
cid & name & type & notnull & dflt\_value & pk \\
\midrule\noalign{}
\endhead
\bottomrule\noalign{}
\endlastfoot
0 & pid & INTEGER & 0 & NA & 0 \\
1 & age & REAL & 0 & NA & 0 \\
2 & fio2 & REAL & 0 & NA & 0 \\
3 & pao2 & REAL & 0 & NA & 0 \\
4 & spo2 & INTEGER & 0 & NA & 0 \\
5 & vent & INTEGER & 0 & NA & 0 \\
6 & gcs\_total & INTEGER & 0 & NA & 0 \\
7 & pupil & TEXT & 0 & NA & 0 \\
8 & platelets & INTEGER & 0 & NA & 0 \\
9 & inr & REAL & 0 & NA & 0 \\
\end{longtable}

\section{Organ Dysfunction Scores}\label{organ-dysfunction-scores}

In each of the following eight sections we present details on the
Phoenix scoring and provide examples of applying the Phoenix scoring in
R, Python, and SQL.

\textbf{Important Notes:}

\begin{itemize}
\item
  Phoenix was developed on a data set of pediatric (age 0 to 18 years)
  non-birth hospitalizations.
\item
  Phoenix was developed assuming that missing data maps to a score of
  zero. In the source code for the R package and Python modules you can
  see that missing values are mapped to ``healthy'' values such that a
  score of 0 will be returned. The example SQL queries do this
  explicitly.

  This means any omitted arguments will map to scores of zero where the
  omitted variable is needed. Thus, the organ dysfunction scores, and
  the composite scores are based on the explicitly used inputs.
\item
  The provided example data set provides all the needed inputs for the
  Phoenix criteria, but requires some processing before scoring. For
  example, the SpO\textsubscript{2}:FiO\textsubscript{2} ratio that is
  needed for the respiratory score is not provided in the
  \texttt{sepsis} data set. The componets, \texttt{spo2} and
  \texttt{fio2}, are part of the data set. The reason for this is to
  highlight an important restriction on the use of the
  SpO\textsubscript{2}:FiO\textsubscript{2} ratio: it is only valid for
  SpO\textsubscript{2} values least than or equal to 97.

  Processing the data and creating specific columns for the needed
  inputs, as seen below, would be preferable in practice, for example,
\end{itemize}

\begin{Shaded}
\begin{Highlighting}[]
\NormalTok{sepsis2 }\OtherTok{\textless{}{-}}
  \FunctionTok{with}\NormalTok{(sepsis,}
    \FunctionTok{data.frame}\NormalTok{(}
        \AttributeTok{pfr =}\NormalTok{ pao2 }\SpecialCharTok{/}\NormalTok{ fio2,}
        \AttributeTok{sfr =} \FunctionTok{ifelse}\NormalTok{(spo2 }\SpecialCharTok{\textless{}=} \DecValTok{97}\NormalTok{, spo2 }\SpecialCharTok{/}\NormalTok{ fio2, }\ConstantTok{NA\_real\_}\NormalTok{),}
        \AttributeTok{imv =}\NormalTok{ vent,}
        \AttributeTok{ors =} \FunctionTok{as.integer}\NormalTok{(fio2 }\SpecialCharTok{\textgreater{}} \FloatTok{0.21}\NormalTok{),}
        \AttributeTok{vasos =}\NormalTok{ dobutamine }\SpecialCharTok{+}\NormalTok{ dopamine }\SpecialCharTok{+}\NormalTok{ epinephrine }\SpecialCharTok{+}\NormalTok{ milrinone }\SpecialCharTok{+}\NormalTok{ norepinephrine }\SpecialCharTok{+}\NormalTok{ vasopressin,}
        \AttributeTok{lactate =}\NormalTok{ lactate,}
        \AttributeTok{age =}\NormalTok{ age,}
        \AttributeTok{map =} \FunctionTok{map}\NormalTok{(}\AttributeTok{sbp =}\NormalTok{ sbp, }\AttributeTok{dbp =}\NormalTok{ dbp),}
        \AttributeTok{plts =}\NormalTok{ platelets,}
        \AttributeTok{inr =}\NormalTok{ inr,}
        \AttributeTok{ddimer =}\NormalTok{ d\_dimer,}
        \AttributeTok{fib =}\NormalTok{ fibrinogen,}
        \AttributeTok{gcs =}\NormalTok{ gcs\_total,}
        \AttributeTok{pupils =} \FunctionTok{as.integer}\NormalTok{(pupil }\SpecialCharTok{==} \StringTok{"both{-}fixed"}\NormalTok{),}
        \AttributeTok{glucose =}\NormalTok{ glucose,}
        \AttributeTok{anc =}\NormalTok{ anc,}
        \AttributeTok{alc =}\NormalTok{ alc,}
        \AttributeTok{creatinine =}\NormalTok{ creatinine,}
        \AttributeTok{bilirubin =}\NormalTok{ bilirubin,}
        \AttributeTok{alt =}\NormalTok{ alt)}
\NormalTok{  )}
\end{Highlighting}
\end{Shaded}

The examples that follow will use the \texttt{sepsis} data set to
highlight the data assumptions.

\subsection{Respiratory}\label{respiratory}

The Phoenix scoring for respiratory dysfunction is based on the pSOFA
{[}2{]} criteria.

\subsubsection{Inputs:}\label{inputs}

\begin{itemize}
\item
  \texttt{pf\_ratio}: is the ratio of PaO\textsubscript{2} (partial
  pressure or oxygen in arterial blood, units of mmHg) to
  FiO\textsubscript{2} (fraction of inspiratory oxygen, values expected
  to be between 0.21 for room air, to 1.00). Gathering the
  PaO\textsubscript{2} is an invasive procedure.
\item
  \texttt{sf\_ratio}: The SpO\textsubscript{2} (pulse oximetry) to
  FiO\textsubscript{2} ratio is a non-invasive surrogate for the PF
  ratio. Important note: during the development of the Phoenix criteria
  SF ratios were only valid to consider if the SpO\textsubscript{2} is
  \(\leq 97\). Expected value range for SpO\textsubscript{2} is 0 to
  100.
\item
  \texttt{imv}: Invasive mechanical ventilation. This is an integer
  valued indicator variable: 0 = not intubated; 1 = intubated.
\item
  \texttt{other\_respiratory\_support}: Any oxygen support, e.g.,
  high-flow, non-invasive positive pressure, or IMV. This can be
  inferred if FiO\textsubscript{2} exceeds 0.21.
\end{itemize}

\subsubsection{Phoenix Scoring}\label{phoenix-scoring}

The respiratory score is based on the
PaO\textsubscript{2}:FiO\textsubscript{2} ratio or the
SpO\textsubscript{2}:FiO\textsubscript{2} ratio.
PaO\textsubscript{2}:FiO\textsubscript{2} is used preferentially over
the SpO\textsubscript{2}:FiO\textsubscript{2} with the worst possible
score used. That is, if a patient would have 1 point based on
PaO\textsubscript{2}:FiO\textsubscript{2} but 2 points based on
SpO\textsubscript{2}:FiO\textsubscript{2}, then the score is 2 points.

\begin{longtable}[]{@{}
  >{\raggedright\arraybackslash}p{(\columnwidth - 8\tabcolsep) * \real{0.2941}}
  >{\raggedright\arraybackslash}p{(\columnwidth - 8\tabcolsep) * \real{0.1765}}
  >{\raggedright\arraybackslash}p{(\columnwidth - 8\tabcolsep) * \real{0.1765}}
  >{\raggedright\arraybackslash}p{(\columnwidth - 8\tabcolsep) * \real{0.1765}}
  >{\raggedright\arraybackslash}p{(\columnwidth - 8\tabcolsep) * \real{0.1765}}@{}}
\toprule\noalign{}
\begin{minipage}[b]{\linewidth}\raggedright
\end{minipage} & \begin{minipage}[b]{\linewidth}\raggedright
0 Points
\end{minipage} & \begin{minipage}[b]{\linewidth}\raggedright
1 Point
\end{minipage} & \begin{minipage}[b]{\linewidth}\raggedright
2 Points
\end{minipage} & \begin{minipage}[b]{\linewidth}\raggedright
3 Points
\end{minipage} \\
\midrule\noalign{}
\endhead
\bottomrule\noalign{}
\endlastfoot
& & Any respiratory support & IMV & IMV \\
~PaO\textsubscript{2}:FiO\textsubscript{2} & \(\geq\) 400 & \textless{}
400 & \textless{} 200 & \textless{} 100 \\
~SpO\textsubscript{2}:FiO\textsubscript{2} & \(\geq\) 292 & \textless{}
292 & \textless{} 220 & \textless{} 148 \\
\end{longtable}

\includegraphics{supplement_files/figure-pdf/respiratory_score_graphic-1.pdf}

\subsubsection{R}\label{r-2}

The \texttt{phoenix\_respiratory} call in R returns an integer vector of
scores. The inputs are expected to be vectors of equal length and can be
explicitly defined from a data set or short handed by passing the data
set in via the \texttt{data} argument as shown below.

\begin{Shaded}
\begin{Highlighting}[]
\FunctionTok{phoenix\_respiratory}\NormalTok{(}
  \AttributeTok{pf\_ratio =}\NormalTok{ pao2 }\SpecialCharTok{/}\NormalTok{ fio2,}
  \AttributeTok{sf\_ratio =} \FunctionTok{ifelse}\NormalTok{(spo2 }\SpecialCharTok{\textless{}=} \DecValTok{97}\NormalTok{, spo2 }\SpecialCharTok{/}\NormalTok{ fio2, }\ConstantTok{NA\_real\_}\NormalTok{),}
  \AttributeTok{imv =}\NormalTok{ vent,}
  \AttributeTok{other\_respiratory\_support =} \FunctionTok{as.integer}\NormalTok{(fio2 }\SpecialCharTok{\textgreater{}} \FloatTok{0.21}\NormalTok{),}
  \AttributeTok{data =}\NormalTok{ sepsis}
\NormalTok{)}
\DocumentationTok{\#\#  [1] 0 3 3 0 0 3 3 0 3 3 3 1 0 2 3 0 2 3 2 0}
\end{Highlighting}
\end{Shaded}

The above explicitly shows the data assumptions for the
SpO\textsubscript{2}:FiO\textsubscript{2} ratio (\texttt{sf\_ratio}) and
\texttt{other\_respiratory\_support}. If the data has be processed with
these assumptions already, or if you only have the
SpO\textsubscript{2}:FiO\textsubscript{2} ratio, the inputs can be
simplified as

\begin{Shaded}
\begin{Highlighting}[]
\FunctionTok{phoenix\_respiratory}\NormalTok{(}
  \AttributeTok{pf\_ratio =}\NormalTok{ pfr,}
  \AttributeTok{sf\_ratio =}\NormalTok{ sfr,}
  \AttributeTok{imv =}\NormalTok{ imv,}
  \AttributeTok{other\_respiratory\_support =}\NormalTok{ ors,}
  \AttributeTok{data =}\NormalTok{ sepsis2}
\NormalTok{)}
\DocumentationTok{\#\#  [1] 0 3 3 0 0 3 3 0 3 3 3 1 0 2 3 0 2 3 2 0}
\end{Highlighting}
\end{Shaded}

\subsubsection{Python}\label{python-2}

The Python implementation is similar to the implementation in R. The
notable difference in the lack of the \texttt{data} argument.

\begin{Shaded}
\begin{Highlighting}[]
\NormalTok{py\_resp }\OperatorTok{=}\NormalTok{ phx.phoenix\_respiratory(}
\NormalTok{    pf\_ratio }\OperatorTok{=}\NormalTok{ sepsis[}\StringTok{"pao2"}\NormalTok{] }\OperatorTok{/}\NormalTok{ sepsis[}\StringTok{"fio2"}\NormalTok{],}
\NormalTok{    sf\_ratio }\OperatorTok{=}\NormalTok{ np.where(sepsis[}\StringTok{"spo2"}\NormalTok{] }\OperatorTok{\textless{}=} \DecValTok{97}\NormalTok{, sepsis[}\StringTok{"spo2"}\NormalTok{] }\OperatorTok{/}\NormalTok{ sepsis[}\StringTok{"fio2"}\NormalTok{], np.nan),}
\NormalTok{    imv      }\OperatorTok{=}\NormalTok{ sepsis[}\StringTok{"vent"}\NormalTok{],}
\NormalTok{    other\_respiratory\_support }\OperatorTok{=}\NormalTok{ (sepsis[}\StringTok{"fio2"}\NormalTok{] }\OperatorTok{\textgreater{}} \FloatTok{0.21}\NormalTok{).astype(}\BuiltInTok{int}\NormalTok{).to\_numpy()}
\NormalTok{)}
\BuiltInTok{print}\NormalTok{(}\BuiltInTok{type}\NormalTok{(py\_resp))}
\CommentTok{\#\# \textless{}class \textquotesingle{}numpy.ndarray\textquotesingle{}\textgreater{}}
\BuiltInTok{print}\NormalTok{(py\_resp)}
\CommentTok{\#\# [0 3 3 0 0 3 3 0 3 3 3 1 0 2 3 0 2 3 2 0]}
\end{Highlighting}
\end{Shaded}

\subsubsection{SQLite}\label{sqlite}

For the SQLite example we start by constructing the needed variables
form the \texttt{sepsis} table and then building the score. The results
will be stored in new table so we can use them again when assessing the
Phoenix and Phoenix-8 scores.

\begin{Shaded}
\begin{Highlighting}[]
\KeywordTok{CREATE} \KeywordTok{TABLE} \ControlFlowTok{IF} \KeywordTok{NOT} \KeywordTok{EXISTS}\NormalTok{ respiratory }\KeywordTok{AS}
  \KeywordTok{SELECT}
\NormalTok{    pid,}
\NormalTok{    (}
\NormalTok{      imv }\OperatorTok{*}\NormalTok{ (}
\NormalTok{        IIF(pfr }\OperatorTok{\textless{}} \DecValTok{100} \KeywordTok{OR}\NormalTok{ sfr }\OperatorTok{\textless{}} \DecValTok{148}\NormalTok{, }\DecValTok{1}\NormalTok{, }\DecValTok{0}\NormalTok{) }\OperatorTok{+}\NormalTok{ IIF(pfr }\OperatorTok{\textless{}} \DecValTok{200} \KeywordTok{OR}\NormalTok{ sfr }\OperatorTok{\textless{}} \DecValTok{220}\NormalTok{, }\DecValTok{1}\NormalTok{, }\DecValTok{0}\NormalTok{)}
\NormalTok{      ) }\OperatorTok{+}
\NormalTok{      other\_respiratory\_support }\OperatorTok{*}\NormalTok{ IIF(pfr }\OperatorTok{\textless{}} \DecValTok{400} \KeywordTok{OR}\NormalTok{ sfr }\OperatorTok{\textless{}} \DecValTok{292}\NormalTok{, }\DecValTok{1}\NormalTok{, }\DecValTok{0}\NormalTok{)}
\NormalTok{    ) }\KeywordTok{AS}\NormalTok{ phoenix\_respiratory\_score}
  \KeywordTok{FROM}
\NormalTok{  (}
    \KeywordTok{SELECT}
\NormalTok{      pid,}
      \FunctionTok{COALESCE}\NormalTok{(pao2 }\OperatorTok{/}\NormalTok{ fio2, }\DecValTok{500}\NormalTok{) }\KeywordTok{AS}\NormalTok{ pfr,}
      \FunctionTok{COALESCE}\NormalTok{(IIF(spo2 }\OperatorTok{\textless{}=} \DecValTok{97}\NormalTok{, spo2 }\OperatorTok{/}\NormalTok{ fio2, }\DecValTok{500}\NormalTok{), }\DecValTok{500}\NormalTok{) }\KeywordTok{AS}\NormalTok{ sfr,}
      \FunctionTok{COALESCE}\NormalTok{(vent, }\DecValTok{0}\NormalTok{) }\KeywordTok{AS}\NormalTok{ imv,}
\NormalTok{      IIF(fio2 }\OperatorTok{\textgreater{}} \FloatTok{0.21} \KeywordTok{OR}\NormalTok{ vent }\OperatorTok{=} \DecValTok{1}\NormalTok{, }\DecValTok{1}\NormalTok{, }\DecValTok{0}\NormalTok{) }\KeywordTok{AS}\NormalTok{ other\_respiratory\_support}
    \KeywordTok{FROM}\NormalTok{ sepsis}
\NormalTok{  );}
\end{Highlighting}
\end{Shaded}

\begin{Shaded}
\begin{Highlighting}[]
\KeywordTok{SELECT} \OperatorTok{*} \KeywordTok{from}\NormalTok{ respiratory}
\end{Highlighting}
\end{Shaded}

\begin{longtable}[]{@{}lr@{}}
\caption{Displaying records 1 - 10}\tabularnewline
\toprule\noalign{}
pid & phoenix\_respiratory\_score \\
\midrule\noalign{}
\endfirsthead
\toprule\noalign{}
pid & phoenix\_respiratory\_score \\
\midrule\noalign{}
\endhead
\bottomrule\noalign{}
\endlastfoot
1 & 0 \\
2 & 3 \\
3 & 3 \\
4 & 0 \\
5 & 0 \\
6 & 3 \\
7 & 3 \\
8 & 0 \\
9 & 3 \\
10 & 3 \\
\end{longtable}

\subsection{Cardiovascular}\label{cardiovascular}

There are three components of the Phoenix scoring for cardiovascular
dysfunction. The number of systemic vasoactive medications is a
modification of the vasoactive-inotropic score{[}3{]} (VIS) and the
cardiovascular score from the Pediatric Logistic Organ Dysfunction-2
(PELOD-2){[}4{]}.

\subsubsection{Inputs}\label{inputs-1}

\begin{itemize}
\item
  \texttt{vasoactives}: is an integer count of the number of systemic
  vasoactive medications the patient is currently receiving. During
  development of the Phoenix criteria it was found that just the count
  of the medications was sufficient to be useful, the dosage was not
  needed. There were six medications considered: dobutamine, dopamine,
  epinephrine, milrinone, norepinephrine, and vasopressin. Again, it is
  systemic use of the medication that is important. For example, an
  injection of epinephrine to halt an allergic reaction would not count,
  whereas having an epinephrine drip to treat hypotension or bradycardia
  would count.
\item
  \texttt{lactate}: level of lactate in the blood, measured in mmol/L
\item
  \texttt{age}: in months
\item
  \texttt{map}: mean arterial pressure (mmHg). During development of the
  Phoenix criteria, map, and blood pressure values in general, obtained
  from arterial measures were used preferentially over values obtained
  from cuffs. Reported values were used preferentially over calculated
  values. If you need to calculate the map use DBP + (1/3) * (SBP - DBP)
  where DBP is diastolic blood pressure (mmHg) and SBP is systolic blood
  pressure (mmHg).
\end{itemize}

\subsubsection{Phoenix Scoring}\label{phoenix-scoring-1}

The Phoenix score is the sum of the vasoactive medications score,
lactate score, and mean MAP score for a total score ranging between 0
and 6 points.

\begin{longtable}[]{@{}
  >{\raggedright\arraybackslash}p{(\columnwidth - 6\tabcolsep) * \real{0.3571}}
  >{\raggedright\arraybackslash}p{(\columnwidth - 6\tabcolsep) * \real{0.2143}}
  >{\raggedright\arraybackslash}p{(\columnwidth - 6\tabcolsep) * \real{0.2143}}
  >{\raggedright\arraybackslash}p{(\columnwidth - 6\tabcolsep) * \real{0.2143}}@{}}
\toprule\noalign{}
\begin{minipage}[b]{\linewidth}\raggedright
\end{minipage} & \begin{minipage}[b]{\linewidth}\raggedright
0 Points
\end{minipage} & \begin{minipage}[b]{\linewidth}\raggedright
1 Point
\end{minipage} & \begin{minipage}[b]{\linewidth}\raggedright
2 Points
\end{minipage} \\
\midrule\noalign{}
\endhead
\bottomrule\noalign{}
\endlastfoot
~ Systemic Vasoactive Medications & No medications & 1 medication & 2 or
more medications \\
~ Lactate (mmol/L) & \(<\) 5 & 5 \(\leq\) Lactate \(<\) 11 & \(\geq\)
11 \\
~ Age (months) adjusted MAP (mmHg) & & & \\
~~ 0 \(\leq\) Age \(<\) 1 & \(\geq\) 31 & 17 \(\leq\) MAP \(<\) 31 &
\(<\) 17 \\
~~ 1 \(\leq\) Age \(<\) 12 & \(\geq\) 39 & 25 \(\leq\) MAP \(<\) 39 &
\(<\) 25 \\
~~ 12 \(\leq\) Age \(<\) 24 & \(\geq\) 44 & 31 \(\leq\) MAP \(<\) 44 &
\(<\) 31 \\
~~ 24 \(\leq\) Age \(<\) 60 & \(\geq\) 45 & 32 \(\leq\) MAP \(<\) 45 &
\(<\) 32 \\
~~ 60 \(\leq\) Age \(<\) 144 & \(\geq\) 49 & 36 \(\leq\) MAP \(<\) 49 &
\(<\) 36 \\
~~ 144 \(\leq\) Age \(<\) 216 & \(\geq\) 52 & 38 \(\leq\) MAP \(<\) 52 &
\(<\) 38 \\
\end{longtable}

\includegraphics{supplement_files/figure-pdf/cardiovascular_score_graphic-1.pdf}

\subsubsection{R}\label{r-3}

For this example data set we can specify a sum for the
\texttt{vasoactives} since each of the columns are indicators. The mean
arterial pressure is estimated based on the reported systolic and
diastolic pressures. The R package \texttt{phoenix} provides the
function \texttt{map} to simplify the estimate:
\(MAP = \frac{2}{3}DBP + \frac{1}{3}SBP\).

\begin{Shaded}
\begin{Highlighting}[]
\FunctionTok{phoenix\_cardiovascular}\NormalTok{(}
  \AttributeTok{vasoactives =}\NormalTok{ dobutamine }\SpecialCharTok{+}\NormalTok{ dopamine }\SpecialCharTok{+}\NormalTok{ epinephrine }\SpecialCharTok{+}
\NormalTok{                milrinone }\SpecialCharTok{+}\NormalTok{ norepinephrine }\SpecialCharTok{+}\NormalTok{ vasopressin,}
  \AttributeTok{lactate =}\NormalTok{ lactate,}
  \AttributeTok{age =}\NormalTok{ age,}
  \AttributeTok{map =} \FunctionTok{map}\NormalTok{(sbp, dbp),}
  \AttributeTok{data =}\NormalTok{ sepsis}
\NormalTok{)}
\DocumentationTok{\#\#  [1] 2 2 1 0 0 1 4 0 3 0 3 0 0 2 3 2 2 2 2 1}
\end{Highlighting}
\end{Shaded}

\subsubsection{Python}\label{python-3}

As with the R package, the Python module provides the function
\texttt{map} to simplify the estimation of the mean arterial pressure.

\begin{Shaded}
\begin{Highlighting}[]
\NormalTok{py\_card }\OperatorTok{=}\NormalTok{ phx.phoenix\_cardiovascular(}
\NormalTok{    vasoactives }\OperatorTok{=}\NormalTok{ sepsis[}\StringTok{"dobutamine"}\NormalTok{] }\OperatorTok{+}\NormalTok{ sepsis[}\StringTok{"dopamine"}\NormalTok{] }\OperatorTok{+}\NormalTok{ sepsis[}\StringTok{"epinephrine"}\NormalTok{] }\OperatorTok{+}
\NormalTok{                  sepsis[}\StringTok{"milrinone"}\NormalTok{] }\OperatorTok{+}\NormalTok{ sepsis[}\StringTok{"norepinephrine"}\NormalTok{] }\OperatorTok{+}\NormalTok{ sepsis[}\StringTok{"vasopressin"}\NormalTok{],}
\NormalTok{    lactate }\OperatorTok{=}\NormalTok{ sepsis[}\StringTok{"lactate"}\NormalTok{],}
\NormalTok{    age }\OperatorTok{=}\NormalTok{ sepsis[}\StringTok{"age"}\NormalTok{],}
    \BuiltInTok{map} \OperatorTok{=}\NormalTok{ phx.}\BuiltInTok{map}\NormalTok{(sepsis[}\StringTok{"sbp"}\NormalTok{], sepsis[}\StringTok{"dbp"}\NormalTok{])}
\NormalTok{)}
\BuiltInTok{print}\NormalTok{(}\BuiltInTok{type}\NormalTok{(py\_card))}
\CommentTok{\#\# \textless{}class \textquotesingle{}numpy.ndarray\textquotesingle{}\textgreater{}}
\BuiltInTok{print}\NormalTok{(py\_card)}
\CommentTok{\#\# [2 2 1 0 0 1 4 0 3 0 3 0 0 2 3 2 2 2 2 1]}
\end{Highlighting}
\end{Shaded}

\subsubsection{SQLite}\label{sqlite-1}

For the SQLite example we will create a new table to store the results
which will be used when getting the overall Phoenix and Phoenix-8
scores.

\begin{Shaded}
\begin{Highlighting}[]
\KeywordTok{CREATE} \KeywordTok{TABLE} \ControlFlowTok{IF} \KeywordTok{NOT} \KeywordTok{EXISTS}\NormalTok{ cardiovascular }\KeywordTok{AS}

\KeywordTok{SELECT}
\NormalTok{  pid,}
\NormalTok{  vaso\_points }\OperatorTok{+}\NormalTok{ lactate\_points }\OperatorTok{+}\NormalTok{ map\_points }\KeywordTok{AS}\NormalTok{ phoenix\_cardiovascular\_score}
\KeywordTok{FROM}
\NormalTok{(}
  \KeywordTok{SELECT} \OperatorTok{*}\NormalTok{,}
    \ControlFlowTok{CASE} \ControlFlowTok{WHEN}\NormalTok{ vasos }\OperatorTok{\textgreater{}} \DecValTok{1} \ControlFlowTok{THEN} \DecValTok{2}
         \ControlFlowTok{WHEN}\NormalTok{ vasos }\OperatorTok{\textgreater{}} \DecValTok{0} \ControlFlowTok{THEN} \DecValTok{1}
         \ControlFlowTok{ELSE} \DecValTok{0} \ControlFlowTok{END} \KeywordTok{AS}\NormalTok{ vaso\_points,}
    \ControlFlowTok{CASE} \ControlFlowTok{WHEN}\NormalTok{ lactate }\OperatorTok{\textgreater{}=} \DecValTok{11} \ControlFlowTok{THEN} \DecValTok{2}
         \ControlFlowTok{WHEN}\NormalTok{ lactate }\OperatorTok{\textgreater{}=}  \DecValTok{5} \ControlFlowTok{THEN} \DecValTok{1}
         \ControlFlowTok{ELSE} \DecValTok{0} \ControlFlowTok{END} \KeywordTok{AS}\NormalTok{ lactate\_points,}
    \ControlFlowTok{CASE} \ControlFlowTok{WHEN}\NormalTok{ (age }\OperatorTok{\textgreater{}=}   \DecValTok{0} \KeywordTok{AND}\NormalTok{ age }\OperatorTok{\textless{}}    \DecValTok{1}\NormalTok{) }\KeywordTok{AND}\NormalTok{ (map }\OperatorTok{\textless{}} \DecValTok{17}\NormalTok{) }\ControlFlowTok{THEN} \DecValTok{2}
         \ControlFlowTok{WHEN}\NormalTok{ (age }\OperatorTok{\textgreater{}=}   \DecValTok{1} \KeywordTok{AND}\NormalTok{ age }\OperatorTok{\textless{}}   \DecValTok{12}\NormalTok{) }\KeywordTok{AND}\NormalTok{ (map }\OperatorTok{\textless{}} \DecValTok{25}\NormalTok{) }\ControlFlowTok{THEN} \DecValTok{2}
         \ControlFlowTok{WHEN}\NormalTok{ (age }\OperatorTok{\textgreater{}=}  \DecValTok{12} \KeywordTok{AND}\NormalTok{ age }\OperatorTok{\textless{}}   \DecValTok{24}\NormalTok{) }\KeywordTok{AND}\NormalTok{ (map }\OperatorTok{\textless{}} \DecValTok{31}\NormalTok{) }\ControlFlowTok{THEN} \DecValTok{2}
         \ControlFlowTok{WHEN}\NormalTok{ (age }\OperatorTok{\textgreater{}=}  \DecValTok{24} \KeywordTok{AND}\NormalTok{ age }\OperatorTok{\textless{}}   \DecValTok{60}\NormalTok{) }\KeywordTok{AND}\NormalTok{ (map }\OperatorTok{\textless{}} \DecValTok{32}\NormalTok{) }\ControlFlowTok{THEN} \DecValTok{2}
         \ControlFlowTok{WHEN}\NormalTok{ (age }\OperatorTok{\textgreater{}=}  \DecValTok{60} \KeywordTok{AND}\NormalTok{ age }\OperatorTok{\textless{}}  \DecValTok{144}\NormalTok{) }\KeywordTok{AND}\NormalTok{ (map }\OperatorTok{\textless{}} \DecValTok{36}\NormalTok{) }\ControlFlowTok{THEN} \DecValTok{2}
         \ControlFlowTok{WHEN}\NormalTok{ (age }\OperatorTok{\textgreater{}=} \DecValTok{144} \KeywordTok{AND}\NormalTok{ age }\OperatorTok{\textless{}=} \DecValTok{216}\NormalTok{) }\KeywordTok{AND}\NormalTok{ (map }\OperatorTok{\textless{}} \DecValTok{38}\NormalTok{) }\ControlFlowTok{THEN} \DecValTok{2}
         \ControlFlowTok{WHEN}\NormalTok{ (age }\OperatorTok{\textgreater{}=}   \DecValTok{0} \KeywordTok{AND}\NormalTok{ age }\OperatorTok{\textless{}}    \DecValTok{1}\NormalTok{) }\KeywordTok{AND}\NormalTok{ (map }\OperatorTok{\textless{}} \DecValTok{31}\NormalTok{) }\ControlFlowTok{THEN} \DecValTok{1}
         \ControlFlowTok{WHEN}\NormalTok{ (age }\OperatorTok{\textgreater{}=}   \DecValTok{1} \KeywordTok{AND}\NormalTok{ age }\OperatorTok{\textless{}}   \DecValTok{12}\NormalTok{) }\KeywordTok{AND}\NormalTok{ (map }\OperatorTok{\textless{}} \DecValTok{39}\NormalTok{) }\ControlFlowTok{THEN} \DecValTok{1}
         \ControlFlowTok{WHEN}\NormalTok{ (age }\OperatorTok{\textgreater{}=}  \DecValTok{12} \KeywordTok{AND}\NormalTok{ age }\OperatorTok{\textless{}}   \DecValTok{24}\NormalTok{) }\KeywordTok{AND}\NormalTok{ (map }\OperatorTok{\textless{}} \DecValTok{44}\NormalTok{) }\ControlFlowTok{THEN} \DecValTok{1}
         \ControlFlowTok{WHEN}\NormalTok{ (age }\OperatorTok{\textgreater{}=}  \DecValTok{24} \KeywordTok{AND}\NormalTok{ age }\OperatorTok{\textless{}}   \DecValTok{60}\NormalTok{) }\KeywordTok{AND}\NormalTok{ (map }\OperatorTok{\textless{}} \DecValTok{45}\NormalTok{) }\ControlFlowTok{THEN} \DecValTok{1}
         \ControlFlowTok{WHEN}\NormalTok{ (age }\OperatorTok{\textgreater{}=}  \DecValTok{60} \KeywordTok{AND}\NormalTok{ age }\OperatorTok{\textless{}}  \DecValTok{144}\NormalTok{) }\KeywordTok{AND}\NormalTok{ (map }\OperatorTok{\textless{}} \DecValTok{49}\NormalTok{) }\ControlFlowTok{THEN} \DecValTok{1}
         \ControlFlowTok{WHEN}\NormalTok{ (age }\OperatorTok{\textgreater{}=} \DecValTok{144} \KeywordTok{AND}\NormalTok{ age }\OperatorTok{\textless{}=} \DecValTok{216}\NormalTok{) }\KeywordTok{AND}\NormalTok{ (map }\OperatorTok{\textless{}} \DecValTok{52}\NormalTok{) }\ControlFlowTok{THEN} \DecValTok{1}
         \ControlFlowTok{ELSE} \DecValTok{0} \ControlFlowTok{END} \KeywordTok{AS}\NormalTok{ map\_points}
  \KeywordTok{FROM}
\NormalTok{  (}
    \KeywordTok{SELECT}
\NormalTok{      pid,}
      \FunctionTok{COALESCE}\NormalTok{(dobutamine, }\DecValTok{0}\NormalTok{) }\OperatorTok{+}
        \FunctionTok{COALESCE}\NormalTok{(dopamine, }\DecValTok{0}\NormalTok{) }\OperatorTok{+}
        \FunctionTok{COALESCE}\NormalTok{(epinephrine, }\DecValTok{0}\NormalTok{) }\OperatorTok{+}
        \FunctionTok{COALESCE}\NormalTok{(milrinone, }\DecValTok{0}\NormalTok{) }\OperatorTok{+}
        \FunctionTok{COALESCE}\NormalTok{(norepinephrine, }\DecValTok{0}\NormalTok{) }\OperatorTok{+}
        \FunctionTok{COALESCE}\NormalTok{(vasopressin, }\DecValTok{0}\NormalTok{) }\KeywordTok{AS}\NormalTok{ vasos,}
\NormalTok{      lactate,}
\NormalTok{      dbp }\OperatorTok{+}\NormalTok{ (sbp }\OperatorTok{{-}}\NormalTok{ dbp) }\OperatorTok{/} \DecValTok{3} \KeywordTok{AS}\NormalTok{ map,}
\NormalTok{      age}
    \KeywordTok{FROM}\NormalTok{ sepsis}
\NormalTok{  )}
\NormalTok{);}
\end{Highlighting}
\end{Shaded}

\begin{Shaded}
\begin{Highlighting}[]
\KeywordTok{SELECT} \OperatorTok{*} \KeywordTok{FROM}\NormalTok{ cardiovascular;}
\end{Highlighting}
\end{Shaded}

\begin{longtable}[]{@{}lr@{}}
\caption{Displaying records 1 - 10}\tabularnewline
\toprule\noalign{}
pid & phoenix\_cardiovascular\_score \\
\midrule\noalign{}
\endfirsthead
\toprule\noalign{}
pid & phoenix\_cardiovascular\_score \\
\midrule\noalign{}
\endhead
\bottomrule\noalign{}
\endlastfoot
1 & 2 \\
2 & 2 \\
3 & 1 \\
4 & 0 \\
5 & 0 \\
6 & 1 \\
7 & 4 \\
8 & 0 \\
9 & 3 \\
10 & 0 \\
\end{longtable}

\subsection{Coagulation}\label{coagulation}

The Disseminated intravascular coagulation score (DIC){[}5{]} is the
basis for the Phoenix coagulation dysfunction score.

\subsubsection{Inputs:}\label{inputs-2}

\begin{itemize}
\item
  \texttt{platelets}: in units of 1,000 / \(\mu\)L
\item
  \texttt{inr}: international normalized ratio; a metric for time
  require for blood to clot.
\item
  \texttt{d\_dimer}: in units of mg/L FEU
\item
  \texttt{fibrinogen} : in units of mg/dL
\end{itemize}

\subsubsection{Phoenix Scoring}\label{phoenix-scoring-2}

While there are four components to this score, the maximum number of
points assigned is 2.

\begin{longtable}[]{@{}
  >{\raggedright\arraybackslash}p{(\columnwidth - 4\tabcolsep) * \real{0.4545}}
  >{\raggedright\arraybackslash}p{(\columnwidth - 4\tabcolsep) * \real{0.2727}}
  >{\raggedright\arraybackslash}p{(\columnwidth - 4\tabcolsep) * \real{0.2727}}@{}}
\toprule\noalign{}
\begin{minipage}[b]{\linewidth}\raggedright
\end{minipage} & \begin{minipage}[b]{\linewidth}\raggedright
0 Points
\end{minipage} & \begin{minipage}[b]{\linewidth}\raggedright
1 Point
\end{minipage} \\
\midrule\noalign{}
\endhead
\bottomrule\noalign{}
\endlastfoot
~ Platelets (1000/\(mu\)L) & \(\geq\) 100 & \(<\) 100 \\
~ INR & \(\leq\) 1.3 & \(>\) 1.3 \\
~ D-Dimer (mg/L FEU) & \(\leq\) 2 & \(>\) 2 \\
~ Fibrinogen (mg/dL) & \(\geq\) 100 & \(<\) 100 \\
\end{longtable}

\includegraphics{supplement_files/figure-pdf/coagulation_score_graphic-1.pdf}

\subsubsection{R}\label{r-4}

The needed inputs for the coagulation score are as is within the example
data set.

\begin{Shaded}
\begin{Highlighting}[]
\FunctionTok{phoenix\_coagulation}\NormalTok{(}
  \AttributeTok{platelets =}\NormalTok{ platelets,}
  \AttributeTok{inr =}\NormalTok{ inr,}
  \AttributeTok{d\_dimer =}\NormalTok{ d\_dimer,}
  \AttributeTok{fibrinogen =}\NormalTok{ fibrinogen,}
  \AttributeTok{data =}\NormalTok{ sepsis}
\NormalTok{)}
\DocumentationTok{\#\#  [1] 1 1 2 1 0 2 2 1 1 0 1 0 0 1 2 1 1 2 0 1}
\end{Highlighting}
\end{Shaded}

\subsubsection{Python}\label{python-4}

The needed inputs for the coagulation score are as is within the example
data set.

\begin{Shaded}
\begin{Highlighting}[]
\NormalTok{py\_coag }\OperatorTok{=}\NormalTok{ phx.phoenix\_coagulation(}
\NormalTok{    platelets }\OperatorTok{=}\NormalTok{ sepsis[}\StringTok{\textquotesingle{}platelets\textquotesingle{}}\NormalTok{],}
\NormalTok{    inr }\OperatorTok{=}\NormalTok{ sepsis[}\StringTok{\textquotesingle{}inr\textquotesingle{}}\NormalTok{],}
\NormalTok{    d\_dimer }\OperatorTok{=}\NormalTok{ sepsis[}\StringTok{\textquotesingle{}d\_dimer\textquotesingle{}}\NormalTok{],}
\NormalTok{    fibrinogen }\OperatorTok{=}\NormalTok{ sepsis[}\StringTok{\textquotesingle{}fibrinogen\textquotesingle{}}\NormalTok{]}
\NormalTok{)}
\BuiltInTok{print}\NormalTok{(}\BuiltInTok{type}\NormalTok{(py\_coag))}
\CommentTok{\#\# \textless{}class \textquotesingle{}numpy.ndarray\textquotesingle{}\textgreater{}}
\BuiltInTok{print}\NormalTok{(py\_coag)}
\CommentTok{\#\# [1 1 2 1 0 2 2 1 1 0 1 0 0 1 2 1 1 2 0 1]}
\end{Highlighting}
\end{Shaded}

\subsubsection{SQLite}\label{sqlite-2}

We will create a coagulation table to use when assessing the Phoenix and
Phoenix-8 scores.

\begin{Shaded}
\begin{Highlighting}[]
\KeywordTok{CREATE} \KeywordTok{TABLE} \ControlFlowTok{IF} \KeywordTok{NOT} \KeywordTok{EXISTS}\NormalTok{ coagulation }\KeywordTok{AS}

\KeywordTok{SELECT}
\NormalTok{  pid,}
  \ControlFlowTok{CASE} \ControlFlowTok{WHEN}\NormalTok{ plts }\OperatorTok{+}\NormalTok{ inr }\OperatorTok{+}\NormalTok{ ddm }\OperatorTok{+}\NormalTok{ fib }\OperatorTok{\textgreater{}=} \DecValTok{2} \ControlFlowTok{THEN} \DecValTok{2}
       \ControlFlowTok{ELSE}\NormalTok{ plts }\OperatorTok{+}\NormalTok{ inr }\OperatorTok{+}\NormalTok{ ddm }\OperatorTok{+}\NormalTok{ fib }\ControlFlowTok{END} \KeywordTok{AS}\NormalTok{ phoenix\_coagulation\_score}
\KeywordTok{FROM}\NormalTok{ (}
  \KeywordTok{SELECT}
\NormalTok{    pid,}
    \ControlFlowTok{CASE} \ControlFlowTok{WHEN}\NormalTok{ platelets }\OperatorTok{\textless{}} \DecValTok{100}  \ControlFlowTok{THEN} \DecValTok{1} \ControlFlowTok{ELSE} \DecValTok{0} \ControlFlowTok{END} \KeywordTok{AS}\NormalTok{ plts,}
    \ControlFlowTok{CASE} \ControlFlowTok{WHEN}\NormalTok{ inr }\OperatorTok{\textgreater{}} \FloatTok{1.3}        \ControlFlowTok{THEN} \DecValTok{1} \ControlFlowTok{ELSE} \DecValTok{0} \ControlFlowTok{END} \KeywordTok{AS}\NormalTok{ inr,}
    \ControlFlowTok{CASE} \ControlFlowTok{WHEN}\NormalTok{ d\_dimer }\OperatorTok{\textgreater{}} \DecValTok{2}      \ControlFlowTok{THEN} \DecValTok{1} \ControlFlowTok{ELSE} \DecValTok{0} \ControlFlowTok{END} \KeywordTok{AS}\NormalTok{ ddm,}
    \ControlFlowTok{CASE} \ControlFlowTok{WHEN}\NormalTok{ fibrinogen }\OperatorTok{\textless{}} \DecValTok{100} \ControlFlowTok{THEN} \DecValTok{1} \ControlFlowTok{ELSE} \DecValTok{0} \ControlFlowTok{END} \KeywordTok{AS}\NormalTok{ fib}
  \KeywordTok{FROM}\NormalTok{ sepsis}
\NormalTok{)}
\end{Highlighting}
\end{Shaded}

\begin{Shaded}
\begin{Highlighting}[]
\KeywordTok{SELECT} \OperatorTok{*} \KeywordTok{FROM}\NormalTok{ coagulation}
\end{Highlighting}
\end{Shaded}

\begin{longtable}[]{@{}lr@{}}
\caption{Displaying records 1 - 10}\tabularnewline
\toprule\noalign{}
pid & phoenix\_coagulation\_score \\
\midrule\noalign{}
\endfirsthead
\toprule\noalign{}
pid & phoenix\_coagulation\_score \\
\midrule\noalign{}
\endhead
\bottomrule\noalign{}
\endlastfoot
1 & 1 \\
2 & 1 \\
3 & 2 \\
4 & 1 \\
5 & 0 \\
6 & 2 \\
7 & 2 \\
8 & 1 \\
9 & 1 \\
10 & 0 \\
\end{longtable}

\subsection{Neurologic}\label{neurologic}

The Phoenix neurologic dysfunction scoring is based on the PELOD-2
{[}4{]} neurologic dysfunction scoring.

\subsubsection{Inputs:}\label{inputs-3}

\begin{itemize}
\item
  \texttt{gcs}: an integer vector for the (total) Glasgow Comma Score.
  The total score is the sum of the eye, verbal, and motor scores. Valid
  gcs values are integers 3, 4, 5, \ldots, 14, 15.
\item
  \texttt{fixed\_pupils}: an integer vector of zeros and ones. 1 =
  bilaterally fixed pupils, 0 otherwise.
\end{itemize}

\subsubsection{Phoenix Scoring}\label{phoenix-scoring-3}

\begin{longtable}[]{@{}
  >{\raggedright\arraybackslash}p{(\columnwidth - 6\tabcolsep) * \real{0.3571}}
  >{\raggedright\arraybackslash}p{(\columnwidth - 6\tabcolsep) * \real{0.2143}}
  >{\raggedright\arraybackslash}p{(\columnwidth - 6\tabcolsep) * \real{0.2143}}
  >{\raggedright\arraybackslash}p{(\columnwidth - 6\tabcolsep) * \real{0.2143}}@{}}
\toprule\noalign{}
\begin{minipage}[b]{\linewidth}\raggedright
\end{minipage} & \begin{minipage}[b]{\linewidth}\raggedright
0 Points
\end{minipage} & \begin{minipage}[b]{\linewidth}\raggedright
1 Point
\end{minipage} & \begin{minipage}[b]{\linewidth}\raggedright
2 Points
\end{minipage} \\
\midrule\noalign{}
\endhead
\bottomrule\noalign{}
\endlastfoot
~ & GCS \(\geq\) 11 & GCS \(\leq\) 10 & Bilaterally fixed pupils \\
\end{longtable}

\subsubsection{R}\label{r-5}

The example data reports pupil reactivity as either ``both-fixed'',
``both-reactive'', or ``\,``. We need only create an indicator on the
fly to build the neurologic score.

\begin{Shaded}
\begin{Highlighting}[]
\FunctionTok{phoenix\_neurologic}\NormalTok{(}
  \AttributeTok{gcs =}\NormalTok{ gcs\_total,}
  \AttributeTok{fixed\_pupils =} \FunctionTok{as.integer}\NormalTok{(pupil }\SpecialCharTok{==} \StringTok{"both{-}fixed"}\NormalTok{),}
  \AttributeTok{data =}\NormalTok{ sepsis}
\NormalTok{)}
\DocumentationTok{\#\#  [1] 0 1 0 0 0 1 0 0 1 1 2 0 0 0 0 0 0 0 0 0}
\end{Highlighting}
\end{Shaded}

\subsubsection{Python}\label{python-5}

\begin{Shaded}
\begin{Highlighting}[]
\NormalTok{py\_neur }\OperatorTok{=}\NormalTok{ phx.phoenix\_neurologic(}
\NormalTok{    gcs }\OperatorTok{=}\NormalTok{ sepsis[}\StringTok{"gcs\_total"}\NormalTok{],}
\NormalTok{    fixed\_pupils }\OperatorTok{=}\NormalTok{ (sepsis[}\StringTok{"pupil"}\NormalTok{] }\OperatorTok{==} \StringTok{"both{-}fixed"}\NormalTok{).astype(}\BuiltInTok{int}\NormalTok{)}
\NormalTok{)}
\BuiltInTok{print}\NormalTok{(}\BuiltInTok{type}\NormalTok{(py\_neur))}
\CommentTok{\#\# \textless{}class \textquotesingle{}numpy.ndarray\textquotesingle{}\textgreater{}}
\BuiltInTok{print}\NormalTok{(py\_neur)}
\CommentTok{\#\# [0 1 0 0 0 1 0 0 1 1 2 0 0 0 0 0 0 0 0 0]}
\end{Highlighting}
\end{Shaded}

\subsubsection{SQLite}\label{sqlite-3}

\begin{Shaded}
\begin{Highlighting}[]
\KeywordTok{CREATE} \KeywordTok{TABLE} \ControlFlowTok{IF} \KeywordTok{NOT} \KeywordTok{EXISTS}\NormalTok{ neurologic }\KeywordTok{AS}

\KeywordTok{SELECT}
\NormalTok{  pid,}
  \ControlFlowTok{CASE} \ControlFlowTok{WHEN}\NormalTok{ fixed\_pupils }\OperatorTok{=} \DecValTok{1} \ControlFlowTok{THEN} \DecValTok{2}
       \ControlFlowTok{WHEN}\NormalTok{ gcs }\OperatorTok{=} \DecValTok{1}          \ControlFlowTok{THEN} \DecValTok{1}
       \ControlFlowTok{ELSE} \DecValTok{0} \ControlFlowTok{END} \KeywordTok{AS}\NormalTok{ phoenix\_neurologic\_score}
\KeywordTok{FROM}\NormalTok{ (}
  \KeywordTok{SELECT}
\NormalTok{    pid,}
    \ControlFlowTok{CASE} \ControlFlowTok{WHEN}\NormalTok{ gcs\_total }\OperatorTok{\textless{}=} \DecValTok{10} \ControlFlowTok{THEN} \DecValTok{1} \ControlFlowTok{ELSE} \DecValTok{0} \ControlFlowTok{END} \KeywordTok{AS}\NormalTok{ gcs,}
    \ControlFlowTok{CASE} \ControlFlowTok{WHEN}\NormalTok{ pupil }\OperatorTok{=} \OtherTok{"both{-}fixed"} \ControlFlowTok{THEN} \DecValTok{1} \ControlFlowTok{ELSE} \DecValTok{0} \ControlFlowTok{END} \KeywordTok{AS}\NormalTok{ fixed\_pupils}
  \KeywordTok{FROM}\NormalTok{ sepsis}
\NormalTok{);}
\end{Highlighting}
\end{Shaded}

\begin{Shaded}
\begin{Highlighting}[]
\KeywordTok{SELECT} \OperatorTok{*} \KeywordTok{FROM}\NormalTok{ neurologic;}
\end{Highlighting}
\end{Shaded}

\begin{longtable}[]{@{}lr@{}}
\caption{Displaying records 1 - 10}\tabularnewline
\toprule\noalign{}
pid & phoenix\_neurologic\_score \\
\midrule\noalign{}
\endfirsthead
\toprule\noalign{}
pid & phoenix\_neurologic\_score \\
\midrule\noalign{}
\endhead
\bottomrule\noalign{}
\endlastfoot
1 & 0 \\
2 & 1 \\
3 & 0 \\
4 & 0 \\
5 & 0 \\
6 & 1 \\
7 & 0 \\
8 & 0 \\
9 & 1 \\
10 & 1 \\
\end{longtable}

\subsection{Endocrine}\label{endocrine}

Endocrine dysfunction is only part of the Phoenix-8 scoring and is based
on a subset of the Pediatric organ dysfunction information update
mandate (PODIUM){[}6{]} endocrine thresholds.

\subsubsection{Inputs:}\label{inputs-4}

\begin{itemize}
\tightlist
\item
  \texttt{glucose}: blood glucose in units of mg/dL
\end{itemize}

\subsubsection{Phoenix-8 Scoring}\label{phoenix-8-scoring}

\begin{longtable}[]{@{}
  >{\raggedright\arraybackslash}p{(\columnwidth - 4\tabcolsep) * \real{0.4545}}
  >{\raggedright\arraybackslash}p{(\columnwidth - 4\tabcolsep) * \real{0.2727}}
  >{\raggedright\arraybackslash}p{(\columnwidth - 4\tabcolsep) * \real{0.2727}}@{}}
\toprule\noalign{}
\begin{minipage}[b]{\linewidth}\raggedright
\end{minipage} & \begin{minipage}[b]{\linewidth}\raggedright
0 Points
\end{minipage} & \begin{minipage}[b]{\linewidth}\raggedright
1 Point
\end{minipage} \\
\midrule\noalign{}
\endhead
\bottomrule\noalign{}
\endlastfoot
~ Blood Glucose (mg/dL) & 50 \(\leq\) Blood Glucose \(\leq\) 150 & \(<\)
50; or \(>\) 150 \\
\end{longtable}

\subsubsection{R}\label{r-6}

\begin{Shaded}
\begin{Highlighting}[]
\FunctionTok{phoenix\_endocrine}\NormalTok{(}\AttributeTok{glucose =}\NormalTok{ glucose, }\AttributeTok{data =}\NormalTok{ sepsis)}
\DocumentationTok{\#\#  [1] 0 0 0 0 0 0 0 0 1 0 1 0 0 0 0 0 0 0 0 1}
\end{Highlighting}
\end{Shaded}

\subsubsection{Python}\label{python-6}

\begin{Shaded}
\begin{Highlighting}[]
\NormalTok{py\_endo }\OperatorTok{=}\NormalTok{ phx.phoenix\_endocrine(sepsis[}\StringTok{"glucose"}\NormalTok{])}
\BuiltInTok{print}\NormalTok{(}\BuiltInTok{type}\NormalTok{(py\_endo))}
\CommentTok{\#\# \textless{}class \textquotesingle{}numpy.ndarray\textquotesingle{}\textgreater{}}
\BuiltInTok{print}\NormalTok{(py\_endo)}
\CommentTok{\#\# [0 0 0 0 0 0 0 0 1 0 1 0 0 0 0 0 0 0 0 1]}
\end{Highlighting}
\end{Shaded}

\subsubsection{SQLite}\label{sqlite-4}

\begin{Shaded}
\begin{Highlighting}[]
\KeywordTok{CREATE} \KeywordTok{TABLE} \ControlFlowTok{IF} \KeywordTok{NOT} \KeywordTok{EXISTS}\NormalTok{ endocrine }\KeywordTok{AS}
\KeywordTok{SELECT}
\NormalTok{  pid,}
  \ControlFlowTok{CASE} \ControlFlowTok{WHEN}\NormalTok{ glucose }\OperatorTok{\textless{}}  \DecValTok{50} \ControlFlowTok{THEN} \DecValTok{1}
       \ControlFlowTok{WHEN}\NormalTok{ glucose }\OperatorTok{\textgreater{}} \DecValTok{150} \ControlFlowTok{THEN} \DecValTok{1}
       \ControlFlowTok{ELSE} \DecValTok{0} \ControlFlowTok{END} \KeywordTok{AS}\NormalTok{ phoenix\_endocrine\_score}
\KeywordTok{FROM}\NormalTok{ sepsis}
\end{Highlighting}
\end{Shaded}

\begin{Shaded}
\begin{Highlighting}[]
\KeywordTok{SELECT} \OperatorTok{*} \KeywordTok{FROM}\NormalTok{ endocrine}
\end{Highlighting}
\end{Shaded}

\begin{longtable}[]{@{}lr@{}}
\caption{Displaying records 1 - 10}\tabularnewline
\toprule\noalign{}
pid & phoenix\_endocrine\_score \\
\midrule\noalign{}
\endfirsthead
\toprule\noalign{}
pid & phoenix\_endocrine\_score \\
\midrule\noalign{}
\endhead
\bottomrule\noalign{}
\endlastfoot
1 & 0 \\
2 & 0 \\
3 & 0 \\
4 & 0 \\
5 & 0 \\
6 & 0 \\
7 & 0 \\
8 & 0 \\
9 & 1 \\
10 & 0 \\
\end{longtable}

\subsection{Immunologic}\label{immunologic}

Immunologic dysfunction is only part of the Phoenix-8 scoring and is
based on a subset of the Pediatric organ dysfunction information update
mandate (PODIUM){[}6{]} immunologic thresholds.

\subsubsection{Inputs:}\label{inputs-5}

\begin{itemize}
\item
  \texttt{anc}: Absolute neutrophil count in units of 1,000 cells per
  cubic millimeter
\item
  \texttt{alc}: absolute lymphocyte count in units of 1,000 cells per
  cubic millimeter
\end{itemize}

\subsubsection{Phoenix-8 Scoring}\label{phoenix-8-scoring-1}

\begin{longtable}[]{@{}
  >{\raggedright\arraybackslash}p{(\columnwidth - 4\tabcolsep) * \real{0.4545}}
  >{\raggedright\arraybackslash}p{(\columnwidth - 4\tabcolsep) * \real{0.2727}}
  >{\raggedright\arraybackslash}p{(\columnwidth - 4\tabcolsep) * \real{0.2727}}@{}}
\toprule\noalign{}
\begin{minipage}[b]{\linewidth}\raggedright
\end{minipage} & \begin{minipage}[b]{\linewidth}\raggedright
0 Points
\end{minipage} & \begin{minipage}[b]{\linewidth}\raggedright
1 Point
\end{minipage} \\
\midrule\noalign{}
\endhead
\bottomrule\noalign{}
\endlastfoot
~ ANC (cells/mm\textsuperscript{3}) & \(\geq\) 500 & \(<\) 500 \\
~ ALC (cells/mm\textsuperscript{3}) & \(\geq\) 1000 & \(<\) 1000 \\
\end{longtable}

\includegraphics{supplement_files/figure-pdf/immunologic_score_graphic-1.pdf}

\subsubsection{R}\label{r-7}

\begin{Shaded}
\begin{Highlighting}[]
\FunctionTok{phoenix\_immunologic}\NormalTok{(}\AttributeTok{anc =}\NormalTok{ anc, }\AttributeTok{alc =}\NormalTok{ alc, }\AttributeTok{data =}\NormalTok{ sepsis)}
\DocumentationTok{\#\#  [1] 0 1 1 1 0 1 0 1 1 1 0 0 0 0 0 1 1 0 1 1}
\end{Highlighting}
\end{Shaded}

\subsubsection{Python}\label{python-7}

\begin{Shaded}
\begin{Highlighting}[]
\NormalTok{py\_immu }\OperatorTok{=}\NormalTok{ phx.phoenix\_immunologic(sepsis[}\StringTok{"anc"}\NormalTok{], sepsis[}\StringTok{"alc"}\NormalTok{])}
\BuiltInTok{print}\NormalTok{(}\BuiltInTok{type}\NormalTok{(py\_immu))}
\CommentTok{\#\# \textless{}class \textquotesingle{}numpy.ndarray\textquotesingle{}\textgreater{}}
\BuiltInTok{print}\NormalTok{(py\_immu)}
\CommentTok{\#\# [0 1 1 1 0 1 0 1 1 1 0 0 0 0 0 1 1 0 1 1]}
\end{Highlighting}
\end{Shaded}

\subsubsection{SQLite}\label{sqlite-5}

\begin{Shaded}
\begin{Highlighting}[]
\KeywordTok{CREATE} \KeywordTok{TABLE} \ControlFlowTok{IF} \KeywordTok{NOT} \KeywordTok{EXISTS}\NormalTok{ immunologic }\KeywordTok{AS}
\KeywordTok{SELECT}
\NormalTok{  pid,}
  \ControlFlowTok{CASE} \ControlFlowTok{WHEN}\NormalTok{ anc }\OperatorTok{\textless{}}  \DecValTok{500} \ControlFlowTok{THEN} \DecValTok{1}
       \ControlFlowTok{WHEN}\NormalTok{ alc }\OperatorTok{\textless{}} \DecValTok{1000} \ControlFlowTok{THEN} \DecValTok{1}
       \ControlFlowTok{ELSE} \DecValTok{0} \ControlFlowTok{END} \KeywordTok{AS}\NormalTok{ phoenix\_immunologic\_score}
\KeywordTok{FROM}\NormalTok{ sepsis}
\end{Highlighting}
\end{Shaded}

\begin{Shaded}
\begin{Highlighting}[]
\KeywordTok{SELECT} \OperatorTok{*} \KeywordTok{FROM}\NormalTok{ immunologic}
\end{Highlighting}
\end{Shaded}

\begin{longtable}[]{@{}lr@{}}
\caption{Displaying records 1 - 10}\tabularnewline
\toprule\noalign{}
pid & phoenix\_immunologic\_score \\
\midrule\noalign{}
\endfirsthead
\toprule\noalign{}
pid & phoenix\_immunologic\_score \\
\midrule\noalign{}
\endhead
\bottomrule\noalign{}
\endlastfoot
1 & 0 \\
2 & 1 \\
3 & 1 \\
4 & 1 \\
5 & 0 \\
6 & 1 \\
7 & 0 \\
8 & 1 \\
9 & 1 \\
10 & 1 \\
\end{longtable}

\subsection{Renal}\label{renal}

The renal dysfunction score is only part of the Phoenix-8 scoring and
comes from pSOFA{[}2{]}.

\subsubsection{Inputs:}\label{inputs-6}

\begin{itemize}
\item
  \texttt{creatinine} in units of mg/dL
\item
  \texttt{age} in months
\end{itemize}

\subsubsection{Phoenix-8 Scoring}\label{phoenix-8-scoring-2}

\begin{longtable}[]{@{}
  >{\raggedright\arraybackslash}p{(\columnwidth - 4\tabcolsep) * \real{0.4545}}
  >{\raggedright\arraybackslash}p{(\columnwidth - 4\tabcolsep) * \real{0.2727}}
  >{\raggedright\arraybackslash}p{(\columnwidth - 4\tabcolsep) * \real{0.2727}}@{}}
\toprule\noalign{}
\begin{minipage}[b]{\linewidth}\raggedright
\end{minipage} & \begin{minipage}[b]{\linewidth}\raggedright
0 Points
\end{minipage} & \begin{minipage}[b]{\linewidth}\raggedright
1 Point
\end{minipage} \\
\midrule\noalign{}
\endhead
\bottomrule\noalign{}
\endlastfoot
~ Age (months) adjusted Creatinine (mg/dL) & & \\
~~ 0 \(\leq\) Age \(<\) 1 & \(<\) 0.8 & \(\geq\) 0.8 \\
~~ 1 \(\leq\) Age \(<\) 12 & \(<\) 0.3 & \(\geq\) 0.3 \\
~~ 12 \(\leq\) Age \(<\) 24 & \(<\) 0.4 & \(\geq\) 0.4 \\
~~ 24 \(\leq\) Age \(<\) 60 & \(<\) 0.6 & \(\geq\) 0.6 \\
~~ 60 \(\leq\) Age \(<\) 144 & \(<\) 0.7 & \(\geq\) 0.7 \\
~~ 144 \(\leq\) Age \(<\) 216 & \(<\) 1.0 & \(\geq\) 1.0 \\
\end{longtable}

\includegraphics{supplement_files/figure-pdf/renal_score_graphic-1.pdf}

\subsubsection{R}\label{r-8}

\begin{Shaded}
\begin{Highlighting}[]
\FunctionTok{phoenix\_renal}\NormalTok{(}\AttributeTok{creatinine =}\NormalTok{ creatinine, }\AttributeTok{age =}\NormalTok{ age, }\AttributeTok{data =}\NormalTok{ sepsis)}
\DocumentationTok{\#\#  [1] 1 0 0 0 0 1 1 0 1 0 1 0 0 0 1 0 0 1 1 0}
\end{Highlighting}
\end{Shaded}

\subsubsection{Python}\label{python-8}

\begin{Shaded}
\begin{Highlighting}[]
\NormalTok{py\_renal }\OperatorTok{=}\NormalTok{ phx.phoenix\_renal(sepsis[}\StringTok{"creatinine"}\NormalTok{], sepsis[}\StringTok{"age"}\NormalTok{])}
\BuiltInTok{print}\NormalTok{(}\BuiltInTok{type}\NormalTok{(py\_renal))}
\CommentTok{\#\# \textless{}class \textquotesingle{}numpy.ndarray\textquotesingle{}\textgreater{}}
\BuiltInTok{print}\NormalTok{(py\_renal)}
\CommentTok{\#\# [1 0 0 0 0 1 1 0 1 0 1 0 0 0 1 0 0 1 1 0]}
\end{Highlighting}
\end{Shaded}

\subsubsection{SQLite}\label{sqlite-6}

\begin{Shaded}
\begin{Highlighting}[]
\KeywordTok{CREATE} \KeywordTok{TABLE} \ControlFlowTok{IF} \KeywordTok{NOT} \KeywordTok{EXISTS}\NormalTok{ renal }\KeywordTok{AS}
\KeywordTok{SELECT}
\NormalTok{  pid,}
  \ControlFlowTok{CASE} \ControlFlowTok{WHEN}\NormalTok{ (age }\OperatorTok{\textgreater{}=}   \DecValTok{0} \KeywordTok{AND}\NormalTok{ age }\OperatorTok{\textless{}}    \DecValTok{1}\NormalTok{) }\KeywordTok{AND}\NormalTok{ (creatinine }\OperatorTok{\textgreater{}=} \FloatTok{0.8}\NormalTok{) }\ControlFlowTok{THEN} \DecValTok{1}
       \ControlFlowTok{WHEN}\NormalTok{ (age }\OperatorTok{\textgreater{}=}   \DecValTok{1} \KeywordTok{AND}\NormalTok{ age }\OperatorTok{\textless{}}   \DecValTok{12}\NormalTok{) }\KeywordTok{AND}\NormalTok{ (creatinine }\OperatorTok{\textgreater{}=} \FloatTok{0.3}\NormalTok{) }\ControlFlowTok{THEN} \DecValTok{1}
       \ControlFlowTok{WHEN}\NormalTok{ (age }\OperatorTok{\textgreater{}=}  \DecValTok{12} \KeywordTok{AND}\NormalTok{ age }\OperatorTok{\textless{}}   \DecValTok{24}\NormalTok{) }\KeywordTok{AND}\NormalTok{ (creatinine }\OperatorTok{\textgreater{}=} \FloatTok{0.4}\NormalTok{) }\ControlFlowTok{THEN} \DecValTok{1}
       \ControlFlowTok{WHEN}\NormalTok{ (age }\OperatorTok{\textgreater{}=}  \DecValTok{24} \KeywordTok{AND}\NormalTok{ age }\OperatorTok{\textless{}}   \DecValTok{60}\NormalTok{) }\KeywordTok{AND}\NormalTok{ (creatinine }\OperatorTok{\textgreater{}=} \FloatTok{0.6}\NormalTok{) }\ControlFlowTok{THEN} \DecValTok{1}
       \ControlFlowTok{WHEN}\NormalTok{ (age }\OperatorTok{\textgreater{}=}  \DecValTok{60} \KeywordTok{AND}\NormalTok{ age }\OperatorTok{\textless{}}  \DecValTok{144}\NormalTok{) }\KeywordTok{AND}\NormalTok{ (creatinine }\OperatorTok{\textgreater{}=} \FloatTok{0.7}\NormalTok{) }\ControlFlowTok{THEN} \DecValTok{1}
       \ControlFlowTok{WHEN}\NormalTok{ (age }\OperatorTok{\textgreater{}=} \DecValTok{144} \KeywordTok{AND}\NormalTok{ age }\OperatorTok{\textless{}=} \DecValTok{216}\NormalTok{) }\KeywordTok{AND}\NormalTok{ (creatinine }\OperatorTok{\textgreater{}=} \FloatTok{1.0}\NormalTok{) }\ControlFlowTok{THEN} \DecValTok{1}
       \ControlFlowTok{ELSE} \DecValTok{0} \ControlFlowTok{END} \KeywordTok{AS}\NormalTok{ phoenix\_renal\_score}
\KeywordTok{FROM}\NormalTok{ sepsis}
\end{Highlighting}
\end{Shaded}

\begin{Shaded}
\begin{Highlighting}[]
\KeywordTok{SELECT} \OperatorTok{*} \KeywordTok{FROM}\NormalTok{ renal}
\end{Highlighting}
\end{Shaded}

\begin{longtable}[]{@{}lr@{}}
\caption{Displaying records 1 - 10}\tabularnewline
\toprule\noalign{}
pid & phoenix\_renal\_score \\
\midrule\noalign{}
\endfirsthead
\toprule\noalign{}
pid & phoenix\_renal\_score \\
\midrule\noalign{}
\endhead
\bottomrule\noalign{}
\endlastfoot
1 & 1 \\
2 & 0 \\
3 & 0 \\
4 & 0 \\
5 & 0 \\
6 & 1 \\
7 & 1 \\
8 & 0 \\
9 & 1 \\
10 & 0 \\
\end{longtable}

\subsection{Hepatic}\label{hepatic}

The hepatic scoring is only part of the Phoenix-8 score and comes from
the IPSCC{[}7{]} criteria.

\subsubsection{Inputs:}\label{inputs-7}

\begin{itemize}
\item
  \texttt{bilirubin}: total bilirubin in units of mg/dL
\item
  \texttt{alt}: alanine aminotransferase in units of mg/dL
\end{itemize}

\subsubsection{Phoenix-8 scoring}\label{phoenix-8-scoring-3}

\begin{longtable}[]{@{}
  >{\raggedright\arraybackslash}p{(\columnwidth - 4\tabcolsep) * \real{0.4545}}
  >{\raggedright\arraybackslash}p{(\columnwidth - 4\tabcolsep) * \real{0.2727}}
  >{\raggedright\arraybackslash}p{(\columnwidth - 4\tabcolsep) * \real{0.2727}}@{}}
\toprule\noalign{}
\begin{minipage}[b]{\linewidth}\raggedright
\end{minipage} & \begin{minipage}[b]{\linewidth}\raggedright
0 Points
\end{minipage} & \begin{minipage}[b]{\linewidth}\raggedright
1 Point
\end{minipage} \\
\midrule\noalign{}
\endhead
\bottomrule\noalign{}
\endlastfoot
~ Total Bilirubin (mg/dL) & \(<\) 4 & \(\geq\) 4 \\
~ ALT (IU/L) & \(\leq\) 102 & \(>\) 102 \\
\end{longtable}

\includegraphics{supplement_files/figure-pdf/hepatic_score_graphic-1.pdf}

\subsubsection{R}\label{r-9}

Calling \texttt{phoenix\_hepatic} will return an integer vector of
scores.

\begin{Shaded}
\begin{Highlighting}[]
\FunctionTok{phoenix\_hepatic}\NormalTok{(}\AttributeTok{bilirubin =}\NormalTok{ bilirubin, }\AttributeTok{alt =}\NormalTok{ alt, }\AttributeTok{data =}\NormalTok{ sepsis)}
\DocumentationTok{\#\#  [1] 0 0 1 1 0 0 1 0 1 0 0 0 0 0 1 0 0 1 0 0}
\end{Highlighting}
\end{Shaded}

\subsubsection{Python}\label{python-9}

Calling \texttt{phoenix\_hepatic} will return an integer valued numpy
array.

\begin{Shaded}
\begin{Highlighting}[]
\NormalTok{py\_hepatic }\OperatorTok{=}\NormalTok{ phx.phoenix\_hepatic(sepsis[}\StringTok{"bilirubin"}\NormalTok{], sepsis[}\StringTok{"alt"}\NormalTok{])}
\BuiltInTok{print}\NormalTok{(}\BuiltInTok{type}\NormalTok{(py\_hepatic))}
\CommentTok{\#\# \textless{}class \textquotesingle{}numpy.ndarray\textquotesingle{}\textgreater{}}
\BuiltInTok{print}\NormalTok{(py\_hepatic)}
\CommentTok{\#\# [0 0 1 1 0 0 1 0 1 0 0 0 0 0 1 0 0 1 0 0]}
\end{Highlighting}
\end{Shaded}

\subsubsection{SQLite}\label{sqlite-7}

\begin{Shaded}
\begin{Highlighting}[]
\KeywordTok{CREATE} \KeywordTok{TABLE} \ControlFlowTok{IF} \KeywordTok{NOT} \KeywordTok{EXISTS}\NormalTok{ hepatic }\KeywordTok{AS}

  \KeywordTok{SELECT}
\NormalTok{    pid,}
    \ControlFlowTok{CASE} \ControlFlowTok{WHEN}\NormalTok{ bilirubin }\OperatorTok{\textgreater{}=} \DecValTok{4} \ControlFlowTok{THEN} \DecValTok{1}
         \ControlFlowTok{WHEN}\NormalTok{ alt }\OperatorTok{\textgreater{}} \DecValTok{102} \ControlFlowTok{THEN} \DecValTok{1}
         \ControlFlowTok{ELSE} \DecValTok{0} \ControlFlowTok{END} \KeywordTok{AS}\NormalTok{ phoenix\_hepatic\_score}
  \KeywordTok{FROM}\NormalTok{ sepsis}
\end{Highlighting}
\end{Shaded}

\begin{Shaded}
\begin{Highlighting}[]
\KeywordTok{SELECT} \OperatorTok{*} \KeywordTok{FROM}\NormalTok{ hepatic;}
\end{Highlighting}
\end{Shaded}

\begin{longtable}[]{@{}lr@{}}
\caption{Displaying records 1 - 10}\tabularnewline
\toprule\noalign{}
pid & phoenix\_hepatic\_score \\
\midrule\noalign{}
\endfirsthead
\toprule\noalign{}
pid & phoenix\_hepatic\_score \\
\midrule\noalign{}
\endhead
\bottomrule\noalign{}
\endlastfoot
1 & 0 \\
2 & 0 \\
3 & 1 \\
4 & 1 \\
5 & 0 \\
6 & 0 \\
7 & 1 \\
8 & 0 \\
9 & 1 \\
10 & 0 \\
\end{longtable}

\section{Phoenix Criteria}\label{phoenix-criteria}

The Phoenix Criteria for sepsis is:

\begin{itemize}
\item
  Suspected (or confirmed) infection: define in development as at least
  one dose of a systemic antimicrobial medication and at least one test
  for an infection ordered. Development of the criteria restricted these
  actions to the first 24 hours of an hospital encounter to minimize
  potential impacts from hospital acquired infections.
\item
  Phoenix Score: the sum of the respiratory, cardiovascular,
  coagulation, and neurologic score.
\item
  Sepsis: A pediatric patient with a suspected (or confirmed) infection
  and a Phoenix Score of at least two points is defined to have sepsis.
\item
  Septic Shock: Sepsis with at least one cardiovascular point.
\end{itemize}

\subsection{R}\label{r-10}

The inputs for the R function \texttt{phoenix} are the same as the
inputs for the four individual organ dysfunction scores. The return form
\texttt{phoenix} is a \texttt{data.frame} with a column for each of the
four organ dysfunction scores, the Phoenix score, and indicators for
sepsis and septic shock.

\begin{Shaded}
\begin{Highlighting}[]
\NormalTok{phoenix\_scores }\OtherTok{\textless{}{-}}
  \FunctionTok{phoenix}\NormalTok{(}
    \CommentTok{\# respiratory}
      \AttributeTok{pf\_ratio =}\NormalTok{ pao2 }\SpecialCharTok{/}\NormalTok{ fio2,}
      \AttributeTok{sf\_ratio =} \FunctionTok{ifelse}\NormalTok{(spo2 }\SpecialCharTok{\textless{}=} \DecValTok{97}\NormalTok{, spo2 }\SpecialCharTok{/}\NormalTok{ fio2, }\ConstantTok{NA\_real\_}\NormalTok{),}
      \AttributeTok{imv =}\NormalTok{ vent,}
      \AttributeTok{other\_respiratory\_support =} \FunctionTok{as.integer}\NormalTok{(fio2 }\SpecialCharTok{\textgreater{}} \FloatTok{0.21}\NormalTok{),}
    \CommentTok{\# cardiovascular}
      \AttributeTok{vasoactives =}\NormalTok{ dobutamine }\SpecialCharTok{+}\NormalTok{ dopamine }\SpecialCharTok{+}\NormalTok{ epinephrine }\SpecialCharTok{+}\NormalTok{ milrinone }\SpecialCharTok{+}
\NormalTok{                    norepinephrine }\SpecialCharTok{+}\NormalTok{ vasopressin,}
      \AttributeTok{lactate =}\NormalTok{ lactate,}
      \AttributeTok{age =}\NormalTok{ age,}
      \AttributeTok{map =}\NormalTok{ dbp }\SpecialCharTok{+}\NormalTok{ (sbp }\SpecialCharTok{{-}}\NormalTok{ dbp)}\SpecialCharTok{/}\DecValTok{3}\NormalTok{,}
    \CommentTok{\# coagulation}
      \AttributeTok{platelets =}\NormalTok{ platelets,}
      \AttributeTok{inr =}\NormalTok{ inr,}
      \AttributeTok{d\_dimer =}\NormalTok{ d\_dimer,}
      \AttributeTok{fibrinogen =}\NormalTok{ fibrinogen,}
    \CommentTok{\# neurologic}
      \AttributeTok{gcs =}\NormalTok{ gcs\_total,}
      \AttributeTok{fixed\_pupils =} \FunctionTok{as.integer}\NormalTok{(pupil }\SpecialCharTok{==} \StringTok{"both{-}fixed"}\NormalTok{),}
    \AttributeTok{data =}\NormalTok{ sepsis}
\NormalTok{  )}

\FunctionTok{str}\NormalTok{(phoenix\_scores)}
\DocumentationTok{\#\# \textquotesingle{}data.frame\textquotesingle{}:    20 obs. of  7 variables:}
\DocumentationTok{\#\#  $ phoenix\_respiratory\_score   : int  0 3 3 0 0 3 3 0 3 3 ...}
\DocumentationTok{\#\#  $ phoenix\_cardiovascular\_score: int  2 2 1 0 0 1 4 0 3 0 ...}
\DocumentationTok{\#\#  $ phoenix\_coagulation\_score   : int  1 1 2 1 0 2 2 1 1 0 ...}
\DocumentationTok{\#\#  $ phoenix\_neurologic\_score    : int  0 1 0 0 0 1 0 0 1 1 ...}
\DocumentationTok{\#\#  $ phoenix\_sepsis\_score        : int  3 7 6 1 0 7 9 1 8 4 ...}
\DocumentationTok{\#\#  $ phoenix\_sepsis              : int  1 1 1 0 0 1 1 0 1 1 ...}
\DocumentationTok{\#\#  $ phoenix\_septic\_shock        : int  1 1 1 0 0 1 1 0 1 0 ...}
\end{Highlighting}
\end{Shaded}

\subsection{Python}\label{python-10}

\begin{Shaded}
\begin{Highlighting}[]
\NormalTok{py\_phoenix\_scores }\OperatorTok{=}\NormalTok{ phx.phoenix(}
    \CommentTok{\# resp}
\NormalTok{    pf\_ratio }\OperatorTok{=}\NormalTok{ sepsis[}\StringTok{"pao2"}\NormalTok{] }\OperatorTok{/}\NormalTok{ sepsis[}\StringTok{"fio2"}\NormalTok{],}
\NormalTok{    sf\_ratio }\OperatorTok{=}\NormalTok{ np.where(sepsis[}\StringTok{"spo2"}\NormalTok{] }\OperatorTok{\textless{}=} \DecValTok{97}\NormalTok{, sepsis[}\StringTok{"spo2"}\NormalTok{] }\OperatorTok{/}\NormalTok{ sepsis[}\StringTok{"fio2"}\NormalTok{], np.nan),}
\NormalTok{    imv      }\OperatorTok{=}\NormalTok{ sepsis[}\StringTok{"vent"}\NormalTok{],}
\NormalTok{    other\_respiratory\_support }\OperatorTok{=}\NormalTok{ (sepsis[}\StringTok{"fio2"}\NormalTok{] }\OperatorTok{\textgreater{}} \FloatTok{0.21}\NormalTok{).astype(}\BuiltInTok{int}\NormalTok{).to\_numpy(),}
    \CommentTok{\# cardio}
\NormalTok{    vasoactives }\OperatorTok{=}\NormalTok{ sepsis[}\StringTok{"dobutamine"}\NormalTok{] }\OperatorTok{+}\NormalTok{ sepsis[}\StringTok{"dopamine"}\NormalTok{] }\OperatorTok{+}
\NormalTok{                  sepsis[}\StringTok{"epinephrine"}\NormalTok{] }\OperatorTok{+}\NormalTok{ sepsis[}\StringTok{"milrinone"}\NormalTok{] }\OperatorTok{+}
\NormalTok{                  sepsis[}\StringTok{"norepinephrine"}\NormalTok{] }\OperatorTok{+}\NormalTok{ sepsis[}\StringTok{"vasopressin"}\NormalTok{],}
\NormalTok{    lactate }\OperatorTok{=}\NormalTok{ sepsis[}\StringTok{"lactate"}\NormalTok{],}
\NormalTok{    age }\OperatorTok{=}\NormalTok{ sepsis[}\StringTok{"age"}\NormalTok{],}
    \BuiltInTok{map} \OperatorTok{=}\NormalTok{ phx.}\BuiltInTok{map}\NormalTok{(sepsis[}\StringTok{"sbp"}\NormalTok{], sepsis[}\StringTok{"dbp"}\NormalTok{]),}
    \CommentTok{\# coag}
\NormalTok{    platelets }\OperatorTok{=}\NormalTok{ sepsis[}\StringTok{\textquotesingle{}platelets\textquotesingle{}}\NormalTok{],}
\NormalTok{    inr }\OperatorTok{=}\NormalTok{ sepsis[}\StringTok{\textquotesingle{}inr\textquotesingle{}}\NormalTok{],}
\NormalTok{    d\_dimer }\OperatorTok{=}\NormalTok{ sepsis[}\StringTok{\textquotesingle{}d\_dimer\textquotesingle{}}\NormalTok{],}
\NormalTok{    fibrinogen }\OperatorTok{=}\NormalTok{ sepsis[}\StringTok{\textquotesingle{}fibrinogen\textquotesingle{}}\NormalTok{],}
    \CommentTok{\# neuro}
\NormalTok{    gcs }\OperatorTok{=}\NormalTok{ sepsis[}\StringTok{"gcs\_total"}\NormalTok{],}
\NormalTok{    fixed\_pupils }\OperatorTok{=}\NormalTok{ (sepsis[}\StringTok{"pupil"}\NormalTok{] }\OperatorTok{==} \StringTok{"both{-}fixed"}\NormalTok{).astype(}\BuiltInTok{int}\NormalTok{),}
\NormalTok{    )}
\BuiltInTok{print}\NormalTok{(py\_phoenix\_scores.info())}
\CommentTok{\#\# \textless{}class \textquotesingle{}pandas.core.frame.DataFrame\textquotesingle{}\textgreater{}}
\CommentTok{\#\# RangeIndex: 20 entries, 0 to 19}
\CommentTok{\#\# Data columns (total 7 columns):}
\CommentTok{\#\#  \#   Column                        Non{-}Null Count  Dtype}
\CommentTok{\#\# {-}{-}{-}  {-}{-}{-}{-}{-}{-}                        {-}{-}{-}{-}{-}{-}{-}{-}{-}{-}{-}{-}{-}{-}  {-}{-}{-}{-}{-}}
\CommentTok{\#\#  0   phoenix\_respiratory\_score     20 non{-}null     int64}
\CommentTok{\#\#  1   phoenix\_cardiovascular\_score  20 non{-}null     int64}
\CommentTok{\#\#  2   phoenix\_coagulation\_score     20 non{-}null     int64}
\CommentTok{\#\#  3   phoenix\_neurologic\_score      20 non{-}null     int64}
\CommentTok{\#\#  4   phoenix\_sepsis\_score          20 non{-}null     int64}
\CommentTok{\#\#  5   phoenix\_sepsis                20 non{-}null     int64}
\CommentTok{\#\#  6   phoenix\_septic\_shock          20 non{-}null     int64}
\CommentTok{\#\# dtypes: int64(7)}
\CommentTok{\#\# memory usage: 1.2 KB}
\CommentTok{\#\# None}
\BuiltInTok{print}\NormalTok{(py\_phoenix\_scores.head())}
\CommentTok{\#\#    phoenix\_respiratory\_score  ...  phoenix\_septic\_shock}
\CommentTok{\#\# 0                          0  ...                     1}
\CommentTok{\#\# 1                          3  ...                     1}
\CommentTok{\#\# 2                          3  ...                     1}
\CommentTok{\#\# 3                          0  ...                     0}
\CommentTok{\#\# 4                          0  ...                     0}
\CommentTok{\#\# }
\CommentTok{\#\# [5 rows x 7 columns]}
\end{Highlighting}
\end{Shaded}

\subsection{SQLite}\label{sqlite-8}

\begin{Shaded}
\begin{Highlighting}[]
\KeywordTok{CREATE} \KeywordTok{TABLE} \ControlFlowTok{IF} \KeywordTok{NOT} \KeywordTok{EXISTS}\NormalTok{ phoenix }\KeywordTok{AS}

\KeywordTok{SELECT}
\NormalTok{  respiratory.pid }\KeywordTok{AS}\NormalTok{ pid,}
  \CommentTok{{-}{-}phoenix\_respiratory\_score,}
  \CommentTok{{-}{-}phoenix\_cardiovascular\_score,}
  \CommentTok{{-}{-}phoenix\_coagulation\_score,}
  \CommentTok{{-}{-}phoenix\_neurologic\_score,}
\NormalTok{  phoenix\_respiratory\_score }\OperatorTok{+}\NormalTok{ phoenix\_cardiovascular\_score }\OperatorTok{+}
\NormalTok{    phoenix\_coagulation\_score }\OperatorTok{+}\NormalTok{ phoenix\_neurologic\_score }\KeywordTok{AS}\NormalTok{ phoenix\_sepsis\_score,}

\NormalTok{  IIF(phoenix\_respiratory\_score }\OperatorTok{+}\NormalTok{ phoenix\_cardiovascular\_score }\OperatorTok{+}
\NormalTok{    phoenix\_coagulation\_score }\OperatorTok{+}\NormalTok{ phoenix\_neurologic\_score }\OperatorTok{\textgreater{}=}\DecValTok{2}\NormalTok{, }\DecValTok{1}\NormalTok{, }\DecValTok{0}\NormalTok{) }\KeywordTok{AS}\NormalTok{ phoenix\_sepsis,}

\NormalTok{  IIF(phoenix\_respiratory\_score }\OperatorTok{+}\NormalTok{ phoenix\_cardiovascular\_score }\OperatorTok{+}
\NormalTok{    phoenix\_coagulation\_score }\OperatorTok{+}\NormalTok{ phoenix\_neurologic\_score }\OperatorTok{\textgreater{}=}\DecValTok{2} \KeywordTok{AND}
\NormalTok{    phoenix\_cardiovascular\_score }\OperatorTok{\textgreater{}} \DecValTok{0}\NormalTok{, }\DecValTok{1}\NormalTok{, }\DecValTok{0}\NormalTok{) }\KeywordTok{AS}\NormalTok{ phoenix\_septic\_shock}

\KeywordTok{FROM}\NormalTok{ respiratory}
\KeywordTok{LEFT} \KeywordTok{JOIN}\NormalTok{ cardiovascular}
\KeywordTok{ON}\NormalTok{ respiratory.pid }\OperatorTok{=}\NormalTok{ cardiovascular.pid}
\KeywordTok{LEFT} \KeywordTok{JOIN}\NormalTok{ coagulation}
\KeywordTok{ON}\NormalTok{ respiratory.pid }\OperatorTok{=}\NormalTok{ coagulation.pid}
\KeywordTok{LEFT} \KeywordTok{JOIN}\NormalTok{ neurologic}
\KeywordTok{ON}\NormalTok{ respiratory.pid }\OperatorTok{=}\NormalTok{ neurologic.pid}
\end{Highlighting}
\end{Shaded}

\begin{Shaded}
\begin{Highlighting}[]
\KeywordTok{SELECT} \OperatorTok{*} \KeywordTok{FROM}\NormalTok{ phoenix}
\end{Highlighting}
\end{Shaded}

\begin{longtable}[]{@{}lrrr@{}}
\caption{Displaying records 1 - 10}\tabularnewline
\toprule\noalign{}
pid & phoenix\_sepsis\_score & phoenix\_sepsis &
phoenix\_septic\_shock \\
\midrule\noalign{}
\endfirsthead
\toprule\noalign{}
pid & phoenix\_sepsis\_score & phoenix\_sepsis &
phoenix\_septic\_shock \\
\midrule\noalign{}
\endhead
\bottomrule\noalign{}
\endlastfoot
1 & 3 & 1 & 1 \\
2 & 7 & 1 & 1 \\
3 & 6 & 1 & 1 \\
4 & 1 & 0 & 0 \\
5 & 0 & 0 & 0 \\
6 & 7 & 1 & 1 \\
7 & 9 & 1 & 1 \\
8 & 1 & 0 & 0 \\
9 & 8 & 1 & 1 \\
10 & 4 & 1 & 0 \\
\end{longtable}

\section{Phoenix-8}\label{phoenix-8}

During development of the Phoenix criteria it was determined that the
four-organ-system Phoenix criteria was sufficient for diagnosing sepsis.
An extended eight-organ-system score, Phoenix-8, was defined and
expected to be primarily used for research.

\subsection{R}\label{r-11}

Calling to \texttt{phoenix8} within R will take the same inputs as the 8
organ dysfunction scoring functions. The return is the same as from
\texttt{phoenix} with additional columns for each of the additional four
organ systems and a Phoenix-8 total score.

\begin{Shaded}
\begin{Highlighting}[]
\NormalTok{phoenix8\_scores }\OtherTok{\textless{}{-}}
  \FunctionTok{phoenix8}\NormalTok{(}
    \CommentTok{\# respiratory}
      \AttributeTok{pf\_ratio =}\NormalTok{ pao2 }\SpecialCharTok{/}\NormalTok{ fio2,}
      \AttributeTok{sf\_ratio =} \FunctionTok{ifelse}\NormalTok{(spo2 }\SpecialCharTok{\textless{}=} \DecValTok{97}\NormalTok{, spo2 }\SpecialCharTok{/}\NormalTok{ fio2, }\ConstantTok{NA\_real\_}\NormalTok{),}
      \AttributeTok{imv =}\NormalTok{ vent,}
      \AttributeTok{other\_respiratory\_support =} \FunctionTok{as.integer}\NormalTok{(fio2 }\SpecialCharTok{\textgreater{}} \FloatTok{0.21}\NormalTok{),}
    \CommentTok{\# cardiovascular}
      \AttributeTok{vasoactives =}\NormalTok{ dobutamine }\SpecialCharTok{+}\NormalTok{ dopamine }\SpecialCharTok{+}\NormalTok{ epinephrine }\SpecialCharTok{+}\NormalTok{ milrinone }\SpecialCharTok{+}
\NormalTok{                    norepinephrine }\SpecialCharTok{+}\NormalTok{ vasopressin,}
      \AttributeTok{lactate =}\NormalTok{ lactate,}
      \AttributeTok{age =}\NormalTok{ age, }\CommentTok{\# Also used in the renal assessment.}
      \AttributeTok{map =}\NormalTok{ dbp }\SpecialCharTok{+}\NormalTok{ (sbp }\SpecialCharTok{{-}}\NormalTok{ dbp)}\SpecialCharTok{/}\DecValTok{3}\NormalTok{,}
    \CommentTok{\# coagulation}
      \AttributeTok{platelets =}\NormalTok{ platelets,}
      \AttributeTok{inr =}\NormalTok{ inr,}
      \AttributeTok{d\_dimer =}\NormalTok{ d\_dimer,}
      \AttributeTok{fibrinogen =}\NormalTok{ fibrinogen,}
    \CommentTok{\# neurologic}
      \AttributeTok{gcs =}\NormalTok{ gcs\_total,}
      \AttributeTok{fixed\_pupils =} \FunctionTok{as.integer}\NormalTok{(pupil }\SpecialCharTok{==} \StringTok{"both{-}fixed"}\NormalTok{),}
    \CommentTok{\# endocrine}
      \AttributeTok{glucose =}\NormalTok{ glucose,}
    \CommentTok{\# immunologic}
      \AttributeTok{anc =}\NormalTok{ anc,}
      \AttributeTok{alc =}\NormalTok{ alc,}
    \CommentTok{\# renal}
      \AttributeTok{creatinine =}\NormalTok{ creatinine,}
      \CommentTok{\# no need to specify age again}
    \CommentTok{\# hepatic}
      \AttributeTok{bilirubin =}\NormalTok{ bilirubin,}
      \AttributeTok{alt =}\NormalTok{ alt,}
    \AttributeTok{data =}\NormalTok{ sepsis}
\NormalTok{  )}

\FunctionTok{str}\NormalTok{(phoenix8\_scores)}
\DocumentationTok{\#\# \textquotesingle{}data.frame\textquotesingle{}:    20 obs. of  12 variables:}
\DocumentationTok{\#\#  $ phoenix\_respiratory\_score   : int  0 3 3 0 0 3 3 0 3 3 ...}
\DocumentationTok{\#\#  $ phoenix\_cardiovascular\_score: int  2 2 1 0 0 1 4 0 3 0 ...}
\DocumentationTok{\#\#  $ phoenix\_coagulation\_score   : int  1 1 2 1 0 2 2 1 1 0 ...}
\DocumentationTok{\#\#  $ phoenix\_neurologic\_score    : int  0 1 0 0 0 1 0 0 1 1 ...}
\DocumentationTok{\#\#  $ phoenix\_sepsis\_score        : int  3 7 6 1 0 7 9 1 8 4 ...}
\DocumentationTok{\#\#  $ phoenix\_sepsis              : int  1 1 1 0 0 1 1 0 1 1 ...}
\DocumentationTok{\#\#  $ phoenix\_septic\_shock        : int  1 1 1 0 0 1 1 0 1 0 ...}
\DocumentationTok{\#\#  $ phoenix\_endocrine\_score     : int  0 0 0 0 0 0 0 0 1 0 ...}
\DocumentationTok{\#\#  $ phoenix\_immunologic\_score   : int  0 1 1 1 0 1 0 1 1 1 ...}
\DocumentationTok{\#\#  $ phoenix\_renal\_score         : int  1 0 0 0 0 1 1 0 1 0 ...}
\DocumentationTok{\#\#  $ phoenix\_hepatic\_score       : int  0 0 1 1 0 0 1 0 1 0 ...}
\DocumentationTok{\#\#  $ phoenix8\_sepsis\_score       : int  4 8 8 3 0 9 11 2 12 5 ...}
\end{Highlighting}
\end{Shaded}

\subsection{Python}\label{python-11}

\begin{Shaded}
\begin{Highlighting}[]
\NormalTok{py\_phoenix8\_scores }\OperatorTok{=}\NormalTok{ phx.phoenix8(}
  \CommentTok{\# resp}
\NormalTok{    pf\_ratio }\OperatorTok{=}\NormalTok{ sepsis[}\StringTok{"pao2"}\NormalTok{] }\OperatorTok{/}\NormalTok{ sepsis[}\StringTok{"fio2"}\NormalTok{],}
\NormalTok{    sf\_ratio }\OperatorTok{=}\NormalTok{ np.where(sepsis[}\StringTok{"spo2"}\NormalTok{] }\OperatorTok{\textless{}=} \DecValTok{97}\NormalTok{, sepsis[}\StringTok{"spo2"}\NormalTok{] }\OperatorTok{/}\NormalTok{ sepsis[}\StringTok{"fio2"}\NormalTok{], np.nan),}
\NormalTok{    imv      }\OperatorTok{=}\NormalTok{ sepsis[}\StringTok{"vent"}\NormalTok{],}
\NormalTok{    other\_respiratory\_support }\OperatorTok{=}\NormalTok{ (sepsis[}\StringTok{"fio2"}\NormalTok{] }\OperatorTok{\textgreater{}} \FloatTok{0.21}\NormalTok{).astype(}\BuiltInTok{int}\NormalTok{).to\_numpy(),}
  \CommentTok{\# card}
\NormalTok{    vasoactives }\OperatorTok{=}\NormalTok{ sepsis[}\StringTok{"dobutamine"}\NormalTok{] }\OperatorTok{+}\NormalTok{ sepsis[}\StringTok{"dopamine"}\NormalTok{] }\OperatorTok{+}
\NormalTok{                  sepsis[}\StringTok{"epinephrine"}\NormalTok{] }\OperatorTok{+}\NormalTok{ sepsis[}\StringTok{"milrinone"}\NormalTok{] }\OperatorTok{+}
\NormalTok{                  sepsis[}\StringTok{"norepinephrine"}\NormalTok{] }\OperatorTok{+}\NormalTok{ sepsis[}\StringTok{"vasopressin"}\NormalTok{],}
\NormalTok{    lactate }\OperatorTok{=}\NormalTok{ sepsis[}\StringTok{"lactate"}\NormalTok{],}
    \BuiltInTok{map} \OperatorTok{=}\NormalTok{ phx.}\BuiltInTok{map}\NormalTok{(sepsis[}\StringTok{"sbp"}\NormalTok{], sepsis[}\StringTok{"dbp"}\NormalTok{]),}
\NormalTok{    age }\OperatorTok{=}\NormalTok{ sepsis[}\StringTok{"age"}\NormalTok{], }\CommentTok{\# also used in renal assessment}
  \CommentTok{\# coag}
\NormalTok{    platelets }\OperatorTok{=}\NormalTok{ sepsis[}\StringTok{\textquotesingle{}platelets\textquotesingle{}}\NormalTok{],}
\NormalTok{    inr }\OperatorTok{=}\NormalTok{ sepsis[}\StringTok{\textquotesingle{}inr\textquotesingle{}}\NormalTok{],}
\NormalTok{    d\_dimer }\OperatorTok{=}\NormalTok{ sepsis[}\StringTok{\textquotesingle{}d\_dimer\textquotesingle{}}\NormalTok{],}
\NormalTok{    fibrinogen }\OperatorTok{=}\NormalTok{ sepsis[}\StringTok{\textquotesingle{}fibrinogen\textquotesingle{}}\NormalTok{],}
  \CommentTok{\# neuro}
\NormalTok{    gcs }\OperatorTok{=}\NormalTok{ sepsis[}\StringTok{"gcs\_total"}\NormalTok{],}
\NormalTok{    fixed\_pupils }\OperatorTok{=}\NormalTok{ (sepsis[}\StringTok{"pupil"}\NormalTok{] }\OperatorTok{==} \StringTok{"both{-}fixed"}\NormalTok{).astype(}\BuiltInTok{int}\NormalTok{),}
  \CommentTok{\# endo}
\NormalTok{    glucose }\OperatorTok{=}\NormalTok{ sepsis[}\StringTok{"glucose"}\NormalTok{],}
  \CommentTok{\# immuno}
\NormalTok{    anc }\OperatorTok{=}\NormalTok{ sepsis[}\StringTok{"anc"}\NormalTok{],}
\NormalTok{    alc }\OperatorTok{=}\NormalTok{ sepsis[}\StringTok{"alc"}\NormalTok{],}
  \CommentTok{\# renal}
\NormalTok{    creatinine }\OperatorTok{=}\NormalTok{ sepsis[}\StringTok{"creatinine"}\NormalTok{],}
    \CommentTok{\# no need to specify age again}
  \CommentTok{\# hep}
\NormalTok{    bilirubin }\OperatorTok{=}\NormalTok{ sepsis[}\StringTok{"bilirubin"}\NormalTok{],}
\NormalTok{    alt }\OperatorTok{=}\NormalTok{ sepsis[}\StringTok{"alt"}\NormalTok{],}
\NormalTok{    )}
\BuiltInTok{print}\NormalTok{(py\_phoenix8\_scores.info())}
\CommentTok{\#\# \textless{}class \textquotesingle{}pandas.core.frame.DataFrame\textquotesingle{}\textgreater{}}
\CommentTok{\#\# RangeIndex: 20 entries, 0 to 19}
\CommentTok{\#\# Data columns (total 12 columns):}
\CommentTok{\#\#  \#   Column                        Non{-}Null Count  Dtype}
\CommentTok{\#\# {-}{-}{-}  {-}{-}{-}{-}{-}{-}                        {-}{-}{-}{-}{-}{-}{-}{-}{-}{-}{-}{-}{-}{-}  {-}{-}{-}{-}{-}}
\CommentTok{\#\#  0   phoenix\_respiratory\_score     20 non{-}null     int64}
\CommentTok{\#\#  1   phoenix\_cardiovascular\_score  20 non{-}null     int64}
\CommentTok{\#\#  2   phoenix\_coagulation\_score     20 non{-}null     int64}
\CommentTok{\#\#  3   phoenix\_neurologic\_score      20 non{-}null     int64}
\CommentTok{\#\#  4   phoenix\_sepsis\_score          20 non{-}null     int64}
\CommentTok{\#\#  5   phoenix\_sepsis                20 non{-}null     int64}
\CommentTok{\#\#  6   phoenix\_septic\_shock          20 non{-}null     int64}
\CommentTok{\#\#  7   phoenix\_endocrine\_score       20 non{-}null     int64}
\CommentTok{\#\#  8   phoenix\_immunologic\_score     20 non{-}null     int64}
\CommentTok{\#\#  9   phoenix\_renal\_score           20 non{-}null     int64}
\CommentTok{\#\#  10  phoenix\_hepatic\_score         20 non{-}null     int64}
\CommentTok{\#\#  11  phoenix8\_score                20 non{-}null     int64}
\CommentTok{\#\# dtypes: int64(12)}
\CommentTok{\#\# memory usage: 2.0 KB}
\CommentTok{\#\# None}
\BuiltInTok{print}\NormalTok{(py\_phoenix8\_scores.head())}
\CommentTok{\#\#    phoenix\_respiratory\_score  ...  phoenix8\_score}
\CommentTok{\#\# 0                          0  ...               4}
\CommentTok{\#\# 1                          3  ...               8}
\CommentTok{\#\# 2                          3  ...               8}
\CommentTok{\#\# 3                          0  ...               3}
\CommentTok{\#\# 4                          0  ...               0}
\CommentTok{\#\# }
\CommentTok{\#\# [5 rows x 12 columns]}
\end{Highlighting}
\end{Shaded}

\subsection{SQLite}\label{sqlite-9}

\begin{Shaded}
\begin{Highlighting}[]
\KeywordTok{CREATE} \KeywordTok{TABLE} \ControlFlowTok{IF} \KeywordTok{NOT} \KeywordTok{EXISTS}\NormalTok{ phoenix8 }\KeywordTok{AS}

\KeywordTok{SELECT}
\NormalTok{  respiratory.pid }\KeywordTok{AS}\NormalTok{ pid,}
  \CommentTok{{-}{-}phoenix\_respiratory\_score,}
  \CommentTok{{-}{-}phoenix\_cardiovascular\_score,}
  \CommentTok{{-}{-}phoenix\_coagulation\_score,}
  \CommentTok{{-}{-}phoenix\_neurologic\_score,}

\NormalTok{  phoenix\_respiratory\_score }\OperatorTok{+}\NormalTok{ phoenix\_cardiovascular\_score }\OperatorTok{+}
\NormalTok{    phoenix\_coagulation\_score }\OperatorTok{+}\NormalTok{ phoenix\_neurologic\_score }\KeywordTok{AS}\NormalTok{ phoenix\_sepsis\_score,}

\NormalTok{  IIF(phoenix\_respiratory\_score }\OperatorTok{+}\NormalTok{ phoenix\_cardiovascular\_score }\OperatorTok{+}
\NormalTok{    phoenix\_coagulation\_score }\OperatorTok{+}\NormalTok{ phoenix\_neurologic\_score }\OperatorTok{\textgreater{}=}\DecValTok{2}\NormalTok{, }\DecValTok{1}\NormalTok{, }\DecValTok{0}\NormalTok{) }\KeywordTok{AS}\NormalTok{ phoenix\_sepsis,}

\NormalTok{  IIF(phoenix\_respiratory\_score }\OperatorTok{+}\NormalTok{ phoenix\_cardiovascular\_score }\OperatorTok{+}
\NormalTok{    phoenix\_coagulation\_score }\OperatorTok{+}\NormalTok{ phoenix\_neurologic\_score }\OperatorTok{\textgreater{}=}\DecValTok{2} \KeywordTok{AND}
\NormalTok{    phoenix\_cardiovascular\_score }\OperatorTok{\textgreater{}} \DecValTok{0}\NormalTok{, }\DecValTok{1}\NormalTok{, }\DecValTok{0}\NormalTok{) }\KeywordTok{AS}\NormalTok{ phoenix\_septic\_shock,}

  \CommentTok{{-}{-}phoenix\_endocrine\_score,}
  \CommentTok{{-}{-}phoenix\_immunologic\_score,}
  \CommentTok{{-}{-}phoenix\_renal\_score,}
  \CommentTok{{-}{-}phoenix\_hepatic\_score,}

\NormalTok{  phoenix\_respiratory\_score }\OperatorTok{+}\NormalTok{ phoenix\_cardiovascular\_score }\OperatorTok{+}
\NormalTok{    phoenix\_coagulation\_score }\OperatorTok{+}\NormalTok{ phoenix\_neurologic\_score  }\OperatorTok{+}
\NormalTok{    phoenix\_endocrine\_score }\OperatorTok{+}\NormalTok{ phoenix\_immunologic\_score }\OperatorTok{+}
\NormalTok{    phoenix\_renal\_score }\OperatorTok{+}\NormalTok{ phoenix\_hepatic\_score }\KeywordTok{AS}\NormalTok{ phoenix8\_sepsis\_score}

\KeywordTok{FROM}\NormalTok{ respiratory}
\KeywordTok{LEFT} \KeywordTok{JOIN}\NormalTok{ cardiovascular}
\KeywordTok{ON}\NormalTok{ respiratory.pid }\OperatorTok{=}\NormalTok{ cardiovascular.pid}
\KeywordTok{LEFT} \KeywordTok{JOIN}\NormalTok{ coagulation}
\KeywordTok{ON}\NormalTok{ respiratory.pid }\OperatorTok{=}\NormalTok{ coagulation.pid}
\KeywordTok{LEFT} \KeywordTok{JOIN}\NormalTok{ neurologic}
\KeywordTok{ON}\NormalTok{ respiratory.pid }\OperatorTok{=}\NormalTok{ neurologic.pid}
\KeywordTok{LEFT} \KeywordTok{JOIN}\NormalTok{ endocrine}
\KeywordTok{ON}\NormalTok{ respiratory.pid }\OperatorTok{=}\NormalTok{ endocrine.pid}
\KeywordTok{LEFT} \KeywordTok{JOIN}\NormalTok{ immunologic}
\KeywordTok{ON}\NormalTok{ respiratory.pid }\OperatorTok{=}\NormalTok{ immunologic.pid}
\KeywordTok{LEFT} \KeywordTok{JOIN}\NormalTok{ renal}
\KeywordTok{ON}\NormalTok{ respiratory.pid }\OperatorTok{=}\NormalTok{ renal.pid}
\KeywordTok{LEFT} \KeywordTok{JOIN}\NormalTok{ hepatic}
\KeywordTok{ON}\NormalTok{ respiratory.pid }\OperatorTok{=}\NormalTok{ hepatic.pid}
\end{Highlighting}
\end{Shaded}

\begin{Shaded}
\begin{Highlighting}[]
\KeywordTok{SELECT} \OperatorTok{*} \KeywordTok{FROM}\NormalTok{ phoenix8}
\end{Highlighting}
\end{Shaded}

\begin{longtable}[]{@{}
  >{\raggedright\arraybackslash}p{(\columnwidth - 8\tabcolsep) * \real{0.0482}}
  >{\raggedleft\arraybackslash}p{(\columnwidth - 8\tabcolsep) * \real{0.2530}}
  >{\raggedleft\arraybackslash}p{(\columnwidth - 8\tabcolsep) * \real{0.1807}}
  >{\raggedleft\arraybackslash}p{(\columnwidth - 8\tabcolsep) * \real{0.2530}}
  >{\raggedleft\arraybackslash}p{(\columnwidth - 8\tabcolsep) * \real{0.2651}}@{}}
\caption{Displaying records 1 - 10}\tabularnewline
\toprule\noalign{}
\begin{minipage}[b]{\linewidth}\raggedright
pid
\end{minipage} & \begin{minipage}[b]{\linewidth}\raggedleft
phoenix\_sepsis\_score
\end{minipage} & \begin{minipage}[b]{\linewidth}\raggedleft
phoenix\_sepsis
\end{minipage} & \begin{minipage}[b]{\linewidth}\raggedleft
phoenix\_septic\_shock
\end{minipage} & \begin{minipage}[b]{\linewidth}\raggedleft
phoenix8\_sepsis\_score
\end{minipage} \\
\midrule\noalign{}
\endfirsthead
\toprule\noalign{}
\begin{minipage}[b]{\linewidth}\raggedright
pid
\end{minipage} & \begin{minipage}[b]{\linewidth}\raggedleft
phoenix\_sepsis\_score
\end{minipage} & \begin{minipage}[b]{\linewidth}\raggedleft
phoenix\_sepsis
\end{minipage} & \begin{minipage}[b]{\linewidth}\raggedleft
phoenix\_septic\_shock
\end{minipage} & \begin{minipage}[b]{\linewidth}\raggedleft
phoenix8\_sepsis\_score
\end{minipage} \\
\midrule\noalign{}
\endhead
\bottomrule\noalign{}
\endlastfoot
1 & 3 & 1 & 1 & 4 \\
2 & 7 & 1 & 1 & 8 \\
3 & 6 & 1 & 1 & 8 \\
4 & 1 & 0 & 0 & 3 \\
5 & 0 & 0 & 0 & 0 \\
6 & 7 & 1 & 1 & 9 \\
7 & 9 & 1 & 1 & 11 \\
8 & 1 & 0 & 0 & 2 \\
9 & 8 & 1 & 1 & 12 \\
10 & 4 & 1 & 0 & 5 \\
\end{longtable}

\section{Clinical Vignettes}\label{clinical-vignettes}

These are taken from the supplemental material of
\href{https://doi.org/10.1001/jama.2024.0196}{{[}8{]}}

\subsection{Clinical Vignette 1}\label{clinical-vignette-1}

A previously healthy 3-year-old girl presents to an emergency department
in Lima, Peru, with a temperature of 39C, tachycardia, and irritability.
Blood pressure with an oscillometric device is 67/32 mmHg (mean arterial
pressure of 43 mmHg). She is given fluid resuscitation per local best
practice guidelines, is started on broad spectrum antibiotics, and blood
and urine cultures are sent. After an hour, she becomes hypotensive
again and she is started on a norepinephrine drip. A complete blood
count reveals leukocytosis, mild anemia, and a platelet count of 95
K/\(\mu\)L.

\emph{Phoenix Sepsis Score:}

\begin{itemize}
\tightlist
\item
  0 respiratory points (no hypoxemia or respiratory support)
\item
  2 cardiovascular points (1 for low mean arterial pressure for age, 1
  for use of a vasoactive medication)
\item
  1 coagulation points (for low platelet count)
\item
  0 neurologic points (irritability would result in a Glasgow Coma Scale
  of approximately 14)
\item
  total = 3 points.
\end{itemize}

\emph{Phoenix Sepsis Criteria:} The patient has suspected infection,
\(\geq 2\) points of the Phoenix Sepsis Score, and \(\geq 1\)
cardiovascular points, so she meets criteria for septic shock.

\begin{Shaded}
\begin{Highlighting}[]
\CommentTok{\# R}
\FunctionTok{phoenix}\NormalTok{(}
  \AttributeTok{vasoactives =} \DecValTok{1}\NormalTok{,  }\CommentTok{\# norepinephrine drip}
  \AttributeTok{map =} \FunctionTok{map}\NormalTok{(}\AttributeTok{sbp =} \DecValTok{67}\NormalTok{, }\AttributeTok{dbp =} \DecValTok{32}\NormalTok{), }\CommentTok{\# 43.667 mmHg}
  \AttributeTok{platelets =} \DecValTok{95}\NormalTok{,}
  \AttributeTok{gcs =} \DecValTok{14}\NormalTok{, }\CommentTok{\# irritability}
  \AttributeTok{age =} \DecValTok{3} \SpecialCharTok{*} \DecValTok{12} \CommentTok{\# expected input for age is in months}
\NormalTok{  ) }\SpecialCharTok{|\textgreater{}}
\FunctionTok{str}\NormalTok{()}
\DocumentationTok{\#\# \textquotesingle{}data.frame\textquotesingle{}:    1 obs. of  7 variables:}
\DocumentationTok{\#\#  $ phoenix\_respiratory\_score   : int 0}
\DocumentationTok{\#\#  $ phoenix\_cardiovascular\_score: int 2}
\DocumentationTok{\#\#  $ phoenix\_coagulation\_score   : int 1}
\DocumentationTok{\#\#  $ phoenix\_neurologic\_score    : int 0}
\DocumentationTok{\#\#  $ phoenix\_sepsis\_score        : int 3}
\DocumentationTok{\#\#  $ phoenix\_sepsis              : int 1}
\DocumentationTok{\#\#  $ phoenix\_septic\_shock        : int 1}
\end{Highlighting}
\end{Shaded}

\begin{Shaded}
\begin{Highlighting}[]
\CommentTok{\# python}
\NormalTok{phx.phoenix(}
\NormalTok{  vasoactives }\OperatorTok{=} \DecValTok{1}\NormalTok{,  }\CommentTok{\# norepinephrine drip}
  \BuiltInTok{map} \OperatorTok{=}\NormalTok{ phx.}\BuiltInTok{map}\NormalTok{(sbp }\OperatorTok{=} \DecValTok{67}\NormalTok{, dbp }\OperatorTok{=} \DecValTok{32}\NormalTok{), }\CommentTok{\# 43.667 mmHg}
\NormalTok{  platelets }\OperatorTok{=} \DecValTok{95}\NormalTok{,}
\NormalTok{  gcs }\OperatorTok{=} \DecValTok{14}\NormalTok{, }\CommentTok{\# irritability}
\NormalTok{  age }\OperatorTok{=} \DecValTok{3} \OperatorTok{*} \DecValTok{12} \CommentTok{\# expected input for age is in months}
\NormalTok{  )}
\CommentTok{\#\#    phoenix\_respiratory\_score  ...  phoenix\_septic\_shock}
\CommentTok{\#\# 0                          0  ...                     1}
\CommentTok{\#\# }
\CommentTok{\#\# [1 rows x 7 columns]}
\end{Highlighting}
\end{Shaded}

\subsection{Clinical Vignette 2}\label{clinical-vignette-2}

A 6-year-old boy with a history of prematurity presents with respiratory
distress to his pediatrician's office in Tucson, Arizona. He is noted to
have a temperature of 38.7C, tachypnea, crackles in the left lower
quadrant on chest auscultation, and an oxygen saturation of 89\% on room
air. He is started on supplemental oxygen and is transported to the
local emergency department via ambulance. In the emergency department, a
chest X-ray shows a consolidation in the left lower lobe and hazy
bilateral lung opacities, so he is started on antibiotics for a
suspected bacterial pneumonia. His respiratory status worsens, and he is
started on non-invasive positive pressure ventilation. While awaiting to
be admitted, his level of consciousness deteriorates rapidly: with
nailbed pressure he only opens his eyes briefly, moans in pain, and
withdraws his hand (Glasgow Coma Scale: 2 for eye response + 2 for
verbal response + 4 for motor response = 8). He is intubated using rapid
sequence induction and placed on a conventional ventilator. During this
time, his lowest mean arterial pressure using a non-invasive
oscillometric device is 52 mmHg and he receives a fluid bolus. He is
then transferred to the pediatric intensive care unit where he requires
a high positive end expiratory pressure and an FiO\textsubscript{2} of
0.45 to achieve an oxygen saturation of 92\% (S/F ratio: 204). Complete
blood count and lactate level reveal a platelet count of 120 K/\(\mu\)L
and a serum lactate of 2.9 mmol/L. Given his platelet count below the
normal reference range, a coagulation panel is sent, which reveals an
INR of 1.7, a D-Dimer of 4.4 mg/L, and a fibrinogen of 120 mg/dL.

\emph{Phoenix Sepsis Score:}

\begin{itemize}
\tightlist
\item
  2 respiratory points (for an S/F ratio \(< 292\) on invasive
  mechanical ventilator)
\item
  0 cardiovascular points (mean arterial pressure \(> 48\) mmHg and
  Lactate level \(< 5\) mmol/L)
\item
  2 coagulation points (for high INR and D-Dimer)
\item
  1 neurologic point (Glasgow Coma Scale \(< 10\))
\item
  total = 5 points.
\end{itemize}

\emph{Phoenix Sepsis Criteria:} The patient has a suspected infection,
\(\geq\) 2 points of the Phoenix Sepsis Score, and 0 cardiovascular
points, so he meets criteria for sepsis.

\begin{Shaded}
\begin{Highlighting}[]
\CommentTok{\# R}
\FunctionTok{phoenix}\NormalTok{(}
  \AttributeTok{gcs =} \DecValTok{2} \SpecialCharTok{+} \DecValTok{2} \SpecialCharTok{+} \DecValTok{4}\NormalTok{, }\CommentTok{\# eye + verbal + motor}
  \AttributeTok{map =} \DecValTok{52}\NormalTok{,}
  \AttributeTok{imv =} \DecValTok{1}\NormalTok{,}
  \AttributeTok{sf\_ratio =} \DecValTok{92} \SpecialCharTok{/} \FloatTok{0.45}\NormalTok{,}
  \AttributeTok{platelets =} \DecValTok{120}\NormalTok{,}
  \AttributeTok{lactate =} \FloatTok{2.9}\NormalTok{,}
  \AttributeTok{inr =} \FloatTok{1.7}\NormalTok{,}
  \AttributeTok{d\_dimer =} \FloatTok{4.4}\NormalTok{,}
  \AttributeTok{fibrinogen =} \DecValTok{120}\NormalTok{)}
\DocumentationTok{\#\#   phoenix\_respiratory\_score phoenix\_cardiovascular\_score}
\DocumentationTok{\#\# 1                         2                            0}
\DocumentationTok{\#\#   phoenix\_coagulation\_score phoenix\_neurologic\_score phoenix\_sepsis\_score}
\DocumentationTok{\#\# 1                         2                        1                    5}
\DocumentationTok{\#\#   phoenix\_sepsis phoenix\_septic\_shock}
\DocumentationTok{\#\# 1              1                    0}
\end{Highlighting}
\end{Shaded}

\begin{Shaded}
\begin{Highlighting}[]
\CommentTok{\# Python}
\NormalTok{phx.phoenix(}
\NormalTok{  gcs }\OperatorTok{=} \DecValTok{2} \OperatorTok{+} \DecValTok{2} \OperatorTok{+} \DecValTok{4}\NormalTok{, }\CommentTok{\# eye + verbal + motor}
  \BuiltInTok{map} \OperatorTok{=} \DecValTok{52}\NormalTok{,}
\NormalTok{  imv }\OperatorTok{=} \DecValTok{1}\NormalTok{,}
\NormalTok{  sf\_ratio }\OperatorTok{=} \DecValTok{92} \OperatorTok{/} \FloatTok{0.45}\NormalTok{,}
\NormalTok{  platelets }\OperatorTok{=} \DecValTok{120}\NormalTok{,}
\NormalTok{  lactate }\OperatorTok{=} \FloatTok{2.9}\NormalTok{,}
\NormalTok{  inr }\OperatorTok{=} \FloatTok{1.7}\NormalTok{,}
\NormalTok{  d\_dimer }\OperatorTok{=} \FloatTok{4.4}\NormalTok{,}
\NormalTok{  fibrinogen }\OperatorTok{=} \DecValTok{120}\NormalTok{)}
\CommentTok{\#\#    phoenix\_respiratory\_score  ...  phoenix\_septic\_shock}
\CommentTok{\#\# 0                          2  ...                     0}
\CommentTok{\#\# }
\CommentTok{\#\# [1 rows x 7 columns]}
\end{Highlighting}
\end{Shaded}

\section*{References}\label{references}
\addcontentsline{toc}{section}{References}

\phantomsection\label{refs}
\begin{CSLReferences}{0}{1}
\bibitem[\citeproctext]{ref-r_remotes}
\CSLLeftMargin{1 }%
\CSLRightInline{Csárdi G, Hester J, Wickham H, \emph{et al.}
\emph{\href{https://CRAN.R-project.org/package=remotes}{Remotes: R
package installation from remote repositories, including 'GitHub'}}.
2024.}

\bibitem[\citeproctext]{ref-matric_2017_psofa}
\CSLLeftMargin{2 }%
\CSLRightInline{Matics TJ, Sanchez-Pinto LN. {Adaptation and Validation
of a Pediatric Sequential Organ Failure Assessment Score and Evaluation
of the Sepsis-3 Definitions in Critically Ill Children}. \emph{JAMA
Pediatrics}. 2017;171:e172352--2. doi:
\href{https://doi.org/10.1001/jamapediatrics.2017.2352}{10.1001/jamapediatrics.2017.2352}}

\bibitem[\citeproctext]{ref-gaies_2010_vis}
\CSLLeftMargin{3 }%
\CSLRightInline{Gaies MG, Gurney JG, Yen AH, \emph{et al.}
Vasoactive--inotropic score as a predictor of morbidity and mortality in
infants after cardiopulmonary bypass. \emph{Pediatric critical care
medicine}. 2010;11:234--8.}

\bibitem[\citeproctext]{ref-dewi_2019_pelod2}
\CSLLeftMargin{4 }%
\CSLRightInline{Dewi W, Christie C, Wardhana A, \emph{et al.} Pediatric
logistic organ dysfunction-2 (pelod-2) score as a model for predicting
mortality in pediatric burn injury. \emph{Annals of Burns and Fire
Disasters}. 2019;32:135.}

\bibitem[\citeproctext]{ref-khemani_2009_dic}
\CSLLeftMargin{5 }%
\CSLRightInline{Khemani RG, Bart RD, Alonzo TA, \emph{et al.}
Disseminated intravascular coagulation score is associated with
mortality for children with shock. \emph{Intensive care medicine}.
2009;35:327--33.}

\bibitem[\citeproctext]{ref-bembea_2022_podium}
\CSLLeftMargin{6 }%
\CSLRightInline{Bembea MM, Agus M, Akcan-Arikan A, \emph{et al.}
Pediatric organ dysfunction information update mandate (PODIUM)
contemporary organ dysfunction criteria: Executive summary.
\emph{Pediatrics}. 2022;149:S1--12.}

\bibitem[\citeproctext]{ref-goldstein_2005_international}
\CSLLeftMargin{7 }%
\CSLRightInline{Goldstein B, Giroir B, Randolph A, \emph{et al.}
International pediatric sepsis consensus conference: Definitions for
sepsis and organ dysfunction in pediatrics. \emph{Pediatric critical
care medicine}. 2005;6:2--8.}

\bibitem[\citeproctext]{ref-sanchezpinto_2024_development}
\CSLLeftMargin{8 }%
\CSLRightInline{Sanchez-Pinto LN, Bennett TD, DeWitt PE, \emph{et al.}
{Development and Validation of the Phoenix Criteria for Pediatric Sepsis
and Septic Shock}. \emph{JAMA}. 2024;331:675--86. doi:
\href{https://doi.org/10.1001/jama.2024.0196}{10.1001/jama.2024.0196}}

\end{CSLReferences}



\end{document}
